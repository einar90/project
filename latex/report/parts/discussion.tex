% -*- root: ../project.tex -*-
\section{Discussion} % (fold)
\label{sec:discussion}

\paragraph{Contour Plots} % (fold)
\label{par:contour_plots}
The asymmetry near the diagonals on the contour plots (Figures~\ref{fig:pressure_drop_contour} and \ref{fig:contours-ect10-mrst10}) is caused by the constraint in the the number of contour lines to be drawn by the plotting library. By inspection of the raw pressure data (shown in Appendix~\ref{sec:data}) we see that it is perfectly symmetric.

The differences in magnitude between the MRST and ECL100 pressures stem from the difference in initial pressure set for the models (higher for ECL100). 
% paragraph contour_plots (end)

\paragraph{Regression Lines} % (fold)
\label{par:regression_lines}
The different slopes observed for the regression lines calculated from the solution of the diffusivity equation and from simulated pressure distribution (Figure~\ref{fig:regression-combined}) is due to the dimensionless form of the diffusivity equation being solved and the parameters used in the simulations. If the parameters $k,h,q$ and $\mu$ were chosen so that $\frac{kh}{q\mu}$ equals unity, the lines would likely overlap. No attempts were made to achieve this, as the essential aspect of the plot is the point where the regression line crosses the line where $p_{i,j}-p_{0,0}=0$. The lines calculated from ECL100 and MRST overlap because $k,h,q$ and $\mu$ have the same values in the simulations.

The strong overlap between lines calculated from solutions and simulations on various grid sizes (Figures~\ref{fig:regression-pcm-grid-sizes}-\ref{fig:regression-mrst-grid-sizes}) indicate that the graphical solution has already converged to a large degree at $M=10$. This is supported by the values calculated from the solution of the discretized diffusivity equation with a larger range of grid sizes than the ones simulated (Table~\ref{tbl:peaceman-results}).
% paragraph regression_lines (end)

\paragraph{Fit With Peaceman's Results} % (fold)
\label{par:fit_with_peaceman_s_results}
The perfect fit between the values calculated by numerical solution of the differential flow equation and the results listed in Peaceman's paper \cite{Peaceman1978Interpretation} is as expected, seeing that the solver was implemented as outlined in one of the paper's appendices.
% paragraph fit_with_peaceman_s_results (end)

The results from applying the regression method on the simulated pressure distribution from the ECL100 and MRST simulators fit well only after ignoring the blocks nearest to the well block. This is due to the error term introduced to the radius matrix by the approximation, which, because it is constant, becomes less significant the further away from the well block we move. Because all the block-pressures in the quadrant lie almost perfectly on the regression line, attempts to account for this error term were abandoned, seeing that even if error was corrected it would not affect the regression line significantly.

As seen from the results, the exact method for calculating the equivalent wellbore radius does not yield correct values. This is likely due to it being derived specifically for pressure distributions generated by applying the five-spot pattern to uniform grids. Additionally, it relies on only two pressure values (unlike the regression method which relies on many), making it much more sensitive to the approximations made in the models. 

\paragraph{Possible Improvements To Simulation Models} % (fold)
\label{par:possible_improvements_to_simulation_models}
A better approximation of the corner-placed well should, in theory, be possible by further reducing the size of the corner-blocks. This would also reduce the magnitude of the error term in the radius matrix. This approach, however, resulted in a very ``concentrated'' pressure distribution; i.e. the pressure gradient was very high  close to the wellbore, but completely flat over most of the surface. This distribution is far from the more even gradient resulting from the numerical solution of the differential flow equation. Thus, the regression approach outlined by Peaceman will not yield correct values.
% paragraph possible_improvements_to_simulation_models (end)
 
% section discussion (end)
