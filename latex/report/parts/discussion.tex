% -*- root: ../project.tex -*-
\section{Discussion} % (fold)
\label{sec:discussion}
The perfect fit between the values calculated by numerical solution of the differential flow equation and the results listed in Peaceman's paper \cite{Peaceman1978Interpretation} is as expected, seeing that the solver was implemented as outlined in one of the paper's appendices.

The results from applying the regression method on the simulated pressure distribution from the ECL100 and MRST simulators fit well only after ignoring the blocks nearest to the wellblock. This is due to the error term introduced to the radius matrix by the approximation, which, because it is constant, becomes less significant the further away from the wellblock we move. Because all the block-pressures in the quadrant lie almost perfectly on the regression line, attempts to account for this error term were abandoned, seeing that even if error was corrected it would not affect the regression line significantly.

A better approximation of the corner-placed well should, in theory, be possible by further reducing the size of the corner block. This would also reduce the magnitude of the error term in the radius matrix. This approach, however, resulted in a very ``concentrated'' pressure distribution; i.e. the pressure gradient was very high  close to the wellbore, but completely flat over most of the surface. This distribution is too far from the more even gradient resulting from the numerical solution of the differential flow equation, and thus the regression approach outlined by Peaceman is not applicable.

As seen from the results, the exact method for calculating the equivalent wellbore radius does not yield correct values. This is likely due to it being derived specifically for pressure distributions generated by the five-spot pattern on uniform grids, in addition to relying on only two pressure values (unlike the regression method which relies on many). This makes it much more sensitive to the approximations made in the models.  
% section discussion (end)
