% -*- root: ../project.tex -*-

\section{Literature Review} % (fold)
\label{sec:literature_review}
The foundations for the well index was laid by Peaceman in 1978 \cite{Aavatsmark2003Well} by deriving his well model utilizing the concept of an equivalent wellbore radius. Peaceman did not use the term ``well index'', but defined the well model which the well index is part of.

Peaceman first derived his expression for the equivalent wellbore radius for one-phase flow under isotropic conditions on a square, uniform two-dimensional grid \cite{Peaceman1978Interpretation}. He later extended it to account for anisotropic permeability and non-square grid blocks \cite{Peaceman1983Interpretation}. Finally, completing the theoretical foundations for two-dimensional models, Peaceman extended it to account for off-center wells and multiple wells within a well block \cite{Peaceman1990Interpretation}. Peaceman also showed that the value $r_{eq}=0.2 \Delta x$ holds for both steady and unsteady state, as well as both compressible and incompressible fluids \cite{Peaceman1978Interpretation}.

In Peaceman's 1993 paper on representation of horizontal wells \cite{Peaceman1993Representation}, he concluded that the equations normally used to calculate the equivalent wellbore radius do not hold for horizontal wells because they're often not isolated, i.e. far from the grid boundaries. He argues that equations derived by Babu and Odeh \cite{Babu1991Relation} should be used to calculate $r_{eq}$, given that the grid is relatively uniform and the medium relatively anisotropic. For the case where it is not, he presents a graphical method for taking this into account. He also presents a new type of ``halfway-centered'' grid that reduces the error, as well as a program that takes into account layered reservoirs.

Peaceman's two-dimensional well model is still used in various weighted formulations in three-dimensional grids, but recently progress has been made on deriving a proper well index for these models. Aavatsmark and Klausen has derived a semi-numerical expression for a well index applicable to three-dimensional grids with homogeneous media, where the well may be oriented in an arbitrary direction \cite{Aavatsmark2003Well}.
% subsection developments (end)

% section literature_review (end)
