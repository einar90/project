% -*- root: ../project.tex -*-

\section{Literature review} % (fold)
\label{sec:literature_review}
The foundations for the well index was laid by Peaceman in 1978 \cite{Aavatsmark2003Well} by deriving his well model utilizing the concept of an equivalent wellbore radius. Peaceman did not use the term ``well index'', but defined the well model which the well index is part of. His well model is stated as
\begin{equation}
    \label{eq:review-model}
    p_{wb} = p_{wf} + \frac{q\mu}{2\pi kh} \ln \frac{r_{eq}}{r_w}
\end{equation}
for single-phase, isotropic conditions \cite{Peaceman1978Interpretation}.

Peaceman first derived his expression the equivalent wellbore radius for one-phase, flow under isotropic conditions on a square, uniform two-dimensional grid \cite{Peaceman1978Interpretation}. He later extended it to account for anisotropic permeability and non-square grid blocks \cite{Peaceman1983Interpretation}. Finally, completing the theoretical foundations for two-dimensional single-well models, Peaceman extended it to account for off-center wells and multiple wells within a well block \cite{Peaceman1990Interpretation}. Peaceman also showed that the value $r_{eq}=0.2$ holds for both steady and unsteady state, as well as both compressible and incompressible fluids \cite{Peaceman1978Interpretation}.

In Peaceman's 1993 paper on representing horizontal wells \cite{Peaceman1993Representation}, he concluded that the equation(s) for equivalent wellbore radius does not hold for horizontal wells due to them often not being isolated i.e. far from the grid boundaries. He argues that equations derived by Babu and Odeh \cite{Babu1991Relation} should be used to calculate $r_{eq}$, given that the grid is relatively uniform and the medium relatively anisotropic. For the case where it is not, he presents a graphical method for taking this into account. He also presents a new type of ``halfway-centered'' grid that reduces the error, as well as a program that takes into account layered reservoirs.

Peaceman's two-dimensional well model is also used in various weighted formulations in three-dimensional, but recently progress has been made on deriving proper well index in three-dimensional. Aavatsmark and Klausen \cite{Aavatsmark2003Well} has derived a seminumerical expression for a well index, applicable for three-dimensional grids with homogenous media, where the well may be oriented in an arbitrary direction. It also takes into account neighboring grid blocks.
% subsection developments (end)

% section literature_review (end)
