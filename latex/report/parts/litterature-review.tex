% -*- root: ../project.tex -*-

\section{Literature review} % (fold)
\label{sec:literature_review}
The foundations for the well index was laid by Peaceman in 1978 \cite{Aavatsmark2003Well} by deriving his well model utilizing the concept of an equivalent well-bore radius. Peaceman did not use the term ``well index'', but defined the well model which the well index is part of. His well model is stated as
\begin{equation}
    \label{eq:review-model}
    p_{wb} = p_{wf} + \frac{q\mu}{2\pi kh} \ln \frac{r_{eq}}{r_w}
\end{equation}
for single-phase, isotropic conditions \cite{Peaceman1978Interpretation}.

Peaceman first derived his model for one-phase flow under isotropic conditions on a square 2D grid \cite{Peaceman1978Interpretation}. He later extended it to account for anisotropic permeability and non-square grid blocks \cite{Peaceman1983Interpretation}. Finally, completing the theoretical foundations for 2D single-well models, Peaceman extended it to account for off-center wells and multiple wells within a well block \cite{Peaceman1990Interpretation}.

Peacemans 2D well model is also used in various weighted formulations in 3D \cite{Aavatsmark2003Well}, but recently progress has been made on deriving proper well index in 3D.

% subsection developments (end)

% section literature_review (end)
