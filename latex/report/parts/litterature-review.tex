% -*- root: ../project.tex -*-

\section{Litterature review} % (fold)
\label{sec:litterature_review}
For simplicity only equations dealing for single-phase, two-dimensional, steady-state flow in a homogeneous reservoir will be discussed in this section.

\subsection{Well Indexes} % (fold)
\label{sub:well_indices}
In numerical reservoir simulation, the \gls{wi}\footnote{The term \gls{cf} is also used by some authors\cite{Peaceman2003New,Schlumberger2013Eclipse}.} is used to relate the flowing well pressure, $p_{wf}$\nomenclature{$p_{wf}$}{flowing well pressure},  to the numerical wellblock pressure, $p_{wb}$\nomenclature{$p_{wb}$}{wellblock pressure}. The well index takes into account the wellblock geometry and the location and orientation of the well segment in the gridblock, as well as rock properties \cite{Peaceman2003New}.

The well index is used when calculating the flow rate into the well. For single-phase flow the it may be written as
\begin{equation}
    q = \frac{WI}{\mu} \left( p_{wb} - p_{wf} \right)
\end{equation}
where $q$\nomenclature{$q$}{volumetric flow rate} is the volumetric flow rate into the well \nomenclature{$\mu$}{viscosity}\nomenclature{$WI$}{well index}\cite{Peaceman2003New}.

For a single well in a two-dimensional reservoir reservoir, Darcy's law for radial one-phase flow can be expressed as
\begin{equation}
    q = \frac{kA}{\mu} \frac{\mathrm{d}p}{\mathrm{d}r}
\end{equation}
which after separating variables, integrating from $p_{wf}$ to $p$ and solving for $p$ gives us
\begin{equation}
    p = p_{wf} + \frac{q\mu}{2\pi kh} \ln{\frac{r}{r_w}}
\end{equation}
where $h$\nomenclature{$h$}{formation thickness} is the formation thickness \nomenclature{$k$}{permeability}\nomenclature{$r$}{radius}\nomenclature{$r_w$}{wellbore radius}\cite{Dake1978Developments}.
% subsection well_indices (end)

\begin{figure}[htbp]
    \centering
    \includegraphics[]{figures/plots/peaceman77-regression.pdf}
    \caption{Plot of numerical solution of pressure plotted agains radius from producing well. Calculations are done using the program in Listing~\ref{lst:peaceman77_solver}. The equation is solved for a $10\times 10$ grid.}
    \label{fig:peaceman77_pressure_vs_radius}
\end{figure}

\begin{table}
    \centering
    \caption{Computed equivalent radius of well-block $r_o$ for various values of M. $\Delta p_D$ is the dimensionless pressure drop between producer and injector, $(p_D)_{M,M} - (p_D)_{o,o}$. $r_o/\Delta x$ (exact) refers to the solution using \hl{equation}; $r_o / \Delta x$ (regression) refers to the point where the regression line crosses the $x$-axis (see Figure~\ref{fig:peaceman77_pressure_vs_radius}).}
    \begin{tabular}{rccc}
        \toprule
        $M$ & $\Delta p_D$ & $r_o/\Delta x$ (exact) & $r_o / \Delta x$ (regression)\\
        \midrule
        3   & 0.78571 & 0.1936 & 0.297 \\
        5   & 0.94346 & 0.1965 & 0.225 \\
        10  & 1.16209 & 0.1978 & 0.212 \\
        15  & 1.29078 & 0.1980 & 0.210 \\
        20  & 1.38222 & 0.1981 & 0.208 \\
        32  & 1.53173 & 0.1981 & 0.207 \\
        50  & 1.67375 & 0.1982 & 0.207 \\
        100 & 1.89436 & 0.1982 & 0.208 \\
        \bottomrule
    \end{tabular}
    \label{tbl:label}
\end{table}

% section litterature_review (end)
