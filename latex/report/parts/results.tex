% -*- root: ../project.tex -*-

\section{Results} % (fold)
\label{sec:results}

\subsection{Equivalent wellbore radius from numerical solution of the differential flow equation} % (fold)
\label{sub:equivalent_wellbore_radius_from_numerical_solution_of_differential_flow_equation}

\subsubsection{Approximate calculation of equivalent radius} % (fold)
\label{ssub:approximate_calculation_of_equivalent_radius}
A contour plot of the solution to the discretized \hl{differential flow equation} on the $10 \times 10$ grid in Figure~\ref{fig:peaceman-grid} is shown in Figure~\ref{fig:pressure_drop_contour}, and a plot of the calculated pressure versus radius is shown in Figure~\ref{fig:peaceman77_pressure_vs_radius}. By extrapolating  the regression line in Figure~\ref{fig:peaceman77_pressure_vs_radius} to the horizontal line $p_{i,j}-p_{0,0}=0$, i.e. where $r=r_{eq}$, we get that
\begin{equation}
    r_{eq} = 0.2 \Delta x
\end{equation}
This result is used for a new interpretation of the wellblock pressure: it is the steady-state flowing pressure at $r=r_{eq}=0.2 \Delta x$ \cite{Peaceman1978Interpretation}.

We can also see from Figure~\ref{fig:peaceman77_pressure_vs_radius} that the pressure calculated for the block adjacent to the wellblock, i.e. at $r/\Delta x=1$, is on the straight line. Thus we can substitute $r=\Delta x$ into the radial flow equation \eqref{eq:dake-radial-pressure} to obtain the equation for the pressure in the adjacent block (using the block numbering shown in Figure~\ref{fig:peaceman-block-numbering})
\begin{equation}
    \label{eq:peaceman-pressure-adjacent}
    p_1 = p_{wb} + \frac{q\mu}{2\pi kh} \ln{\frac{\Delta x}{r_{eq}}}
\end{equation}
We now use equation \eqref{eq:differential-flow-discretized},
\begin{equation}
    \label{eq:peaceman-pressure-adjacent-discrete}
    \frac{kh}{\mu} \left( p_1 + p_2 + p_3 + p_4 -4p_{wb}  \right) = q
\end{equation}
where, by symmetry,
\begin{equation}
    \label{eq:peaceman-adjacent-symmetry}
    p_1=p_2=p_3=p_4.
\end{equation}
Combining equations \eqref{eq:peaceman-pressure-adjacent}, \eqref{eq:peaceman-pressure-adjacent-discrete} and \eqref{eq:peaceman-adjacent-symmetry} gives us
\begin{equation}
    \ln{\frac{\Delta x}{r_{eq}}} = \frac{\pi}{2}
\end{equation}
or
\begin{equation}
    r_{eq} = \Delta x \exp{\left( -\frac{\pi}{2} \right)} = 0.208 \Delta x
\end{equation}
which fits well with the results from the intersection of the regression line for $M\times M$ grids when $M\geq 20$ (see Table~\ref{tbl:peaceman-results}).

\begin{figure}[htbp]
    \centering
    \scalebox{0.7}{%% Creator: Matplotlib, PGF backend
%%
%% To include the figure in your LaTeX document, write
%%   \input{<filename>.pgf}
%%
%% Make sure the required packages are loaded in your preamble
%%   \usepackage{pgf}
%%
%% Figures using additional raster images can only be included by \input if
%% they are in the same directory as the main LaTeX file. For loading figures
%% from other directories you can use the `import` package
%%   \usepackage{import}
%% and then include the figures with
%%   \import{<path to file>}{<filename>.pgf}
%%
%% Matplotlib used the following preamble
%%   \usepackage{fontspec}
%%   \setmainfont{DejaVu Serif}
%%   \setsansfont{DejaVu Sans}
%%   \setmonofont{DejaVu Sans Mono}
%%
\begingroup%
\makeatletter%
\begin{pgfpicture}%
\pgfpathrectangle{\pgfpointorigin}{\pgfqpoint{6.000000in}{5.800000in}}%
\pgfusepath{use as bounding box}%
\begin{pgfscope}%
\pgfsetbuttcap%
\pgfsetroundjoin%
\definecolor{currentfill}{rgb}{1.000000,1.000000,1.000000}%
\pgfsetfillcolor{currentfill}%
\pgfsetlinewidth{0.000000pt}%
\definecolor{currentstroke}{rgb}{1.000000,1.000000,1.000000}%
\pgfsetstrokecolor{currentstroke}%
\pgfsetdash{}{0pt}%
\pgfpathmoveto{\pgfqpoint{0.000000in}{0.000000in}}%
\pgfpathlineto{\pgfqpoint{6.000000in}{0.000000in}}%
\pgfpathlineto{\pgfqpoint{6.000000in}{5.800000in}}%
\pgfpathlineto{\pgfqpoint{0.000000in}{5.800000in}}%
\pgfpathclose%
\pgfusepath{fill}%
\end{pgfscope}%
\begin{pgfscope}%
\pgfsetbuttcap%
\pgfsetroundjoin%
\definecolor{currentfill}{rgb}{1.000000,1.000000,1.000000}%
\pgfsetfillcolor{currentfill}%
\pgfsetlinewidth{0.000000pt}%
\definecolor{currentstroke}{rgb}{0.000000,0.000000,0.000000}%
\pgfsetstrokecolor{currentstroke}%
\pgfsetstrokeopacity{0.000000}%
\pgfsetdash{}{0pt}%
\pgfpathmoveto{\pgfqpoint{0.750000in}{0.580000in}}%
\pgfpathlineto{\pgfqpoint{5.400000in}{0.580000in}}%
\pgfpathlineto{\pgfqpoint{5.400000in}{5.220000in}}%
\pgfpathlineto{\pgfqpoint{0.750000in}{5.220000in}}%
\pgfpathclose%
\pgfusepath{fill}%
\end{pgfscope}%
\begin{pgfscope}%
\pgfpathrectangle{\pgfqpoint{0.750000in}{0.580000in}}{\pgfqpoint{4.650000in}{4.640000in}} %
\pgfusepath{clip}%
\pgfsetbuttcap%
\pgfsetroundjoin%
\pgfsetlinewidth{1.003750pt}%
\definecolor{currentstroke}{rgb}{0.000000,0.000000,0.000000}%
\pgfsetstrokecolor{currentstroke}%
\pgfsetdash{}{0pt}%
\pgfpathmoveto{\pgfqpoint{0.796500in}{0.580000in}}%
\pgfpathlineto{\pgfqpoint{0.750000in}{0.626400in}}%
\pgfusepath{stroke}%
\end{pgfscope}%
\begin{pgfscope}%
\pgfpathrectangle{\pgfqpoint{0.750000in}{0.580000in}}{\pgfqpoint{4.650000in}{4.640000in}} %
\pgfusepath{clip}%
\pgfsetbuttcap%
\pgfsetroundjoin%
\pgfsetlinewidth{1.003750pt}%
\definecolor{currentstroke}{rgb}{0.000000,0.000000,0.000000}%
\pgfsetstrokecolor{currentstroke}%
\pgfsetdash{}{0pt}%
\pgfpathmoveto{\pgfqpoint{0.843000in}{0.580000in}}%
\pgfpathlineto{\pgfqpoint{0.750000in}{0.672800in}}%
\pgfusepath{stroke}%
\end{pgfscope}%
\begin{pgfscope}%
\pgfpathrectangle{\pgfqpoint{0.750000in}{0.580000in}}{\pgfqpoint{4.650000in}{4.640000in}} %
\pgfusepath{clip}%
\pgfsetbuttcap%
\pgfsetroundjoin%
\pgfsetlinewidth{1.003750pt}%
\definecolor{currentstroke}{rgb}{0.000000,0.000000,0.000000}%
\pgfsetstrokecolor{currentstroke}%
\pgfsetdash{}{0pt}%
\pgfpathmoveto{\pgfqpoint{0.889500in}{0.580000in}}%
\pgfpathlineto{\pgfqpoint{0.750000in}{0.719200in}}%
\pgfusepath{stroke}%
\end{pgfscope}%
\begin{pgfscope}%
\pgfpathrectangle{\pgfqpoint{0.750000in}{0.580000in}}{\pgfqpoint{4.650000in}{4.640000in}} %
\pgfusepath{clip}%
\pgfsetbuttcap%
\pgfsetroundjoin%
\pgfsetlinewidth{1.003750pt}%
\definecolor{currentstroke}{rgb}{0.000000,0.000000,0.000000}%
\pgfsetstrokecolor{currentstroke}%
\pgfsetdash{}{0pt}%
\pgfpathmoveto{\pgfqpoint{0.985028in}{0.580000in}}%
\pgfpathlineto{\pgfqpoint{0.898907in}{0.725000in}}%
\pgfpathlineto{\pgfqpoint{0.895313in}{0.728587in}}%
\pgfpathlineto{\pgfqpoint{0.750000in}{0.814522in}}%
\pgfusepath{stroke}%
\end{pgfscope}%
\begin{pgfscope}%
\pgfpathrectangle{\pgfqpoint{0.750000in}{0.580000in}}{\pgfqpoint{4.650000in}{4.640000in}} %
\pgfusepath{clip}%
\pgfsetbuttcap%
\pgfsetroundjoin%
\pgfsetlinewidth{1.003750pt}%
\definecolor{currentstroke}{rgb}{0.000000,0.000000,0.000000}%
\pgfsetstrokecolor{currentstroke}%
\pgfsetdash{}{0pt}%
\pgfpathmoveto{\pgfqpoint{1.120169in}{0.580000in}}%
\pgfpathlineto{\pgfqpoint{1.082185in}{0.706979in}}%
\pgfusepath{stroke}%
\end{pgfscope}%
\begin{pgfscope}%
\pgfpathrectangle{\pgfqpoint{0.750000in}{0.580000in}}{\pgfqpoint{4.650000in}{4.640000in}} %
\pgfusepath{clip}%
\pgfsetbuttcap%
\pgfsetroundjoin%
\pgfsetlinewidth{1.003750pt}%
\definecolor{currentstroke}{rgb}{0.000000,0.000000,0.000000}%
\pgfsetstrokecolor{currentstroke}%
\pgfsetdash{}{0pt}%
\pgfpathmoveto{\pgfqpoint{0.769691in}{0.943508in}}%
\pgfpathlineto{\pgfqpoint{0.750000in}{0.949373in}}%
\pgfusepath{stroke}%
\end{pgfscope}%
\begin{pgfscope}%
\pgfpathrectangle{\pgfqpoint{0.750000in}{0.580000in}}{\pgfqpoint{4.650000in}{4.640000in}} %
\pgfusepath{clip}%
\pgfsetbuttcap%
\pgfsetroundjoin%
\pgfsetlinewidth{1.003750pt}%
\definecolor{currentstroke}{rgb}{0.000000,0.000000,0.000000}%
\pgfsetstrokecolor{currentstroke}%
\pgfsetdash{}{0pt}%
\pgfpathmoveto{\pgfqpoint{1.343465in}{0.580000in}}%
\pgfpathlineto{\pgfqpoint{1.331250in}{0.660923in}}%
\pgfpathlineto{\pgfqpoint{1.322987in}{0.725000in}}%
\pgfpathlineto{\pgfqpoint{1.272558in}{0.845909in}}%
\pgfusepath{stroke}%
\end{pgfscope}%
\begin{pgfscope}%
\pgfpathrectangle{\pgfqpoint{0.750000in}{0.580000in}}{\pgfqpoint{4.650000in}{4.640000in}} %
\pgfusepath{clip}%
\pgfsetbuttcap%
\pgfsetroundjoin%
\pgfsetlinewidth{1.003750pt}%
\definecolor{currentstroke}{rgb}{0.000000,0.000000,0.000000}%
\pgfsetstrokecolor{currentstroke}%
\pgfsetdash{}{0pt}%
\pgfpathmoveto{\pgfqpoint{0.972713in}{1.119611in}}%
\pgfpathlineto{\pgfqpoint{0.895313in}{1.151755in}}%
\pgfpathlineto{\pgfqpoint{0.831098in}{1.160000in}}%
\pgfpathlineto{\pgfqpoint{0.750000in}{1.172189in}}%
\pgfusepath{stroke}%
\end{pgfscope}%
\begin{pgfscope}%
\pgfpathrectangle{\pgfqpoint{0.750000in}{0.580000in}}{\pgfqpoint{4.650000in}{4.640000in}} %
\pgfusepath{clip}%
\pgfsetbuttcap%
\pgfsetroundjoin%
\pgfsetlinewidth{1.003750pt}%
\definecolor{currentstroke}{rgb}{0.000000,0.000000,0.000000}%
\pgfsetstrokecolor{currentstroke}%
\pgfsetdash{}{0pt}%
\pgfpathmoveto{\pgfqpoint{1.727870in}{0.580000in}}%
\pgfpathlineto{\pgfqpoint{1.716705in}{0.725000in}}%
\pgfpathlineto{\pgfqpoint{1.681778in}{0.870000in}}%
\pgfpathlineto{\pgfqpoint{1.621875in}{1.009161in}}%
\pgfpathlineto{\pgfqpoint{1.619343in}{1.015000in}}%
\pgfpathlineto{\pgfqpoint{1.536007in}{1.148595in}}%
\pgfusepath{stroke}%
\end{pgfscope}%
\begin{pgfscope}%
\pgfpathrectangle{\pgfqpoint{0.750000in}{0.580000in}}{\pgfqpoint{4.650000in}{4.640000in}} %
\pgfusepath{clip}%
\pgfsetbuttcap%
\pgfsetroundjoin%
\pgfsetlinewidth{1.003750pt}%
\definecolor{currentstroke}{rgb}{0.000000,0.000000,0.000000}%
\pgfsetstrokecolor{currentstroke}%
\pgfsetdash{}{0pt}%
\pgfpathmoveto{\pgfqpoint{1.229290in}{1.420546in}}%
\pgfpathlineto{\pgfqpoint{1.185938in}{1.447473in}}%
\pgfpathlineto{\pgfqpoint{1.180086in}{1.450000in}}%
\pgfpathlineto{\pgfqpoint{1.040625in}{1.509774in}}%
\pgfpathlineto{\pgfqpoint{0.895313in}{1.544626in}}%
\pgfpathlineto{\pgfqpoint{0.750000in}{1.555767in}}%
\pgfusepath{stroke}%
\end{pgfscope}%
\begin{pgfscope}%
\pgfpathrectangle{\pgfqpoint{0.750000in}{0.580000in}}{\pgfqpoint{4.650000in}{4.640000in}} %
\pgfusepath{clip}%
\pgfsetbuttcap%
\pgfsetroundjoin%
\pgfsetlinewidth{1.003750pt}%
\definecolor{currentstroke}{rgb}{0.000000,0.000000,0.000000}%
\pgfsetstrokecolor{currentstroke}%
\pgfsetdash{}{0pt}%
\pgfpathmoveto{\pgfqpoint{2.367468in}{0.580000in}}%
\pgfpathlineto{\pgfqpoint{2.360119in}{0.725000in}}%
\pgfpathlineto{\pgfqpoint{2.348438in}{0.801647in}}%
\pgfpathlineto{\pgfqpoint{2.338716in}{0.870000in}}%
\pgfpathlineto{\pgfqpoint{2.303583in}{1.015000in}}%
\pgfpathlineto{\pgfqpoint{2.251967in}{1.160000in}}%
\pgfpathlineto{\pgfqpoint{2.203125in}{1.262954in}}%
\pgfpathlineto{\pgfqpoint{2.183112in}{1.305000in}}%
\pgfpathlineto{\pgfqpoint{2.096357in}{1.450000in}}%
\pgfpathlineto{\pgfqpoint{2.057812in}{1.503401in}}%
\pgfpathlineto{\pgfqpoint{1.994649in}{1.585532in}}%
\pgfusepath{stroke}%
\end{pgfscope}%
\begin{pgfscope}%
\pgfpathrectangle{\pgfqpoint{0.750000in}{0.580000in}}{\pgfqpoint{4.650000in}{4.640000in}} %
\pgfusepath{clip}%
\pgfsetbuttcap%
\pgfsetroundjoin%
\pgfsetlinewidth{1.003750pt}%
\definecolor{currentstroke}{rgb}{0.000000,0.000000,0.000000}%
\pgfsetstrokecolor{currentstroke}%
\pgfsetdash{}{0pt}%
\pgfpathmoveto{\pgfqpoint{1.693312in}{1.871277in}}%
\pgfpathlineto{\pgfqpoint{1.675391in}{1.885000in}}%
\pgfpathlineto{\pgfqpoint{1.621875in}{1.923462in}}%
\pgfpathlineto{\pgfqpoint{1.476562in}{2.010030in}}%
\pgfpathlineto{\pgfqpoint{1.434426in}{2.030000in}}%
\pgfpathlineto{\pgfqpoint{1.331250in}{2.078737in}}%
\pgfpathlineto{\pgfqpoint{1.185938in}{2.130242in}}%
\pgfpathlineto{\pgfqpoint{1.040625in}{2.165300in}}%
\pgfpathlineto{\pgfqpoint{0.972124in}{2.175000in}}%
\pgfpathlineto{\pgfqpoint{0.895313in}{2.186657in}}%
\pgfpathlineto{\pgfqpoint{0.750000in}{2.193990in}}%
\pgfusepath{stroke}%
\end{pgfscope}%
\begin{pgfscope}%
\pgfpathrectangle{\pgfqpoint{0.750000in}{0.580000in}}{\pgfqpoint{4.650000in}{4.640000in}} %
\pgfusepath{clip}%
\pgfsetbuttcap%
\pgfsetroundjoin%
\pgfsetlinewidth{1.003750pt}%
\definecolor{currentstroke}{rgb}{0.000000,0.000000,0.000000}%
\pgfsetstrokecolor{currentstroke}%
\pgfsetdash{}{0pt}%
\pgfpathmoveto{\pgfqpoint{3.507498in}{0.580000in}}%
\pgfpathlineto{\pgfqpoint{3.501414in}{0.725000in}}%
\pgfpathlineto{\pgfqpoint{3.483078in}{0.870000in}}%
\pgfpathlineto{\pgfqpoint{3.452246in}{1.015000in}}%
\pgfpathlineto{\pgfqpoint{3.408560in}{1.160000in}}%
\pgfpathlineto{\pgfqpoint{3.365625in}{1.269788in}}%
\pgfpathlineto{\pgfqpoint{3.352621in}{1.305000in}}%
\pgfpathlineto{\pgfqpoint{3.286818in}{1.450000in}}%
\pgfpathlineto{\pgfqpoint{3.220312in}{1.572076in}}%
\pgfpathlineto{\pgfqpoint{3.208278in}{1.595000in}}%
\pgfpathlineto{\pgfqpoint{3.120359in}{1.740000in}}%
\pgfpathlineto{\pgfqpoint{3.075000in}{1.805797in}}%
\pgfpathlineto{\pgfqpoint{3.021241in}{1.885000in}}%
\pgfpathlineto{\pgfqpoint{2.929688in}{2.005566in}}%
\pgfpathlineto{\pgfqpoint{2.926863in}{2.009281in}}%
\pgfusepath{stroke}%
\end{pgfscope}%
\begin{pgfscope}%
\pgfpathrectangle{\pgfqpoint{0.750000in}{0.580000in}}{\pgfqpoint{4.650000in}{4.640000in}} %
\pgfusepath{clip}%
\pgfsetbuttcap%
\pgfsetroundjoin%
\pgfsetlinewidth{1.003750pt}%
\definecolor{currentstroke}{rgb}{0.000000,0.000000,0.000000}%
\pgfsetstrokecolor{currentstroke}%
\pgfsetdash{}{0pt}%
\pgfpathmoveto{\pgfqpoint{2.647505in}{2.331427in}}%
\pgfpathlineto{\pgfqpoint{2.639063in}{2.340189in}}%
\pgfpathlineto{\pgfqpoint{2.513983in}{2.465000in}}%
\pgfpathlineto{\pgfqpoint{2.493750in}{2.484412in}}%
\pgfpathlineto{\pgfqpoint{2.355238in}{2.610000in}}%
\pgfpathlineto{\pgfqpoint{2.348438in}{2.616031in}}%
\pgfpathlineto{\pgfqpoint{2.203125in}{2.736467in}}%
\pgfpathlineto{\pgfqpoint{2.178638in}{2.755000in}}%
\pgfpathlineto{\pgfqpoint{2.057812in}{2.846356in}}%
\pgfpathlineto{\pgfqpoint{1.978439in}{2.900000in}}%
\pgfpathlineto{\pgfqpoint{1.912500in}{2.945261in}}%
\pgfpathlineto{\pgfqpoint{1.767188in}{3.032991in}}%
\pgfpathlineto{\pgfqpoint{1.744214in}{3.045000in}}%
\pgfpathlineto{\pgfqpoint{1.621875in}{3.111363in}}%
\pgfpathlineto{\pgfqpoint{1.476562in}{3.177024in}}%
\pgfpathlineto{\pgfqpoint{1.441274in}{3.190000in}}%
\pgfpathlineto{\pgfqpoint{1.331250in}{3.232843in}}%
\pgfpathlineto{\pgfqpoint{1.185938in}{3.276435in}}%
\pgfpathlineto{\pgfqpoint{1.040625in}{3.307201in}}%
\pgfpathlineto{\pgfqpoint{0.895313in}{3.325497in}}%
\pgfpathlineto{\pgfqpoint{0.750000in}{3.331568in}}%
\pgfusepath{stroke}%
\end{pgfscope}%
\begin{pgfscope}%
\pgfpathrectangle{\pgfqpoint{0.750000in}{0.580000in}}{\pgfqpoint{4.650000in}{4.640000in}} %
\pgfusepath{clip}%
\pgfsetbuttcap%
\pgfsetroundjoin%
\pgfsetlinewidth{1.003750pt}%
\definecolor{currentstroke}{rgb}{0.000000,0.000000,0.000000}%
\pgfsetstrokecolor{currentstroke}%
\pgfsetdash{}{0pt}%
\pgfpathmoveto{\pgfqpoint{2.385449in}{5.220000in}}%
\pgfpathlineto{\pgfqpoint{2.392238in}{5.075000in}}%
\pgfpathlineto{\pgfqpoint{2.400178in}{5.018646in}}%
\pgfusepath{stroke}%
\end{pgfscope}%
\begin{pgfscope}%
\pgfpathrectangle{\pgfqpoint{0.750000in}{0.580000in}}{\pgfqpoint{4.650000in}{4.640000in}} %
\pgfusepath{clip}%
\pgfsetbuttcap%
\pgfsetroundjoin%
\pgfsetlinewidth{1.003750pt}%
\definecolor{currentstroke}{rgb}{0.000000,0.000000,0.000000}%
\pgfsetstrokecolor{currentstroke}%
\pgfsetdash{}{0pt}%
\pgfpathmoveto{\pgfqpoint{2.524786in}{4.565304in}}%
\pgfpathlineto{\pgfqpoint{2.552717in}{4.495000in}}%
\pgfpathlineto{\pgfqpoint{2.624035in}{4.350000in}}%
\pgfpathlineto{\pgfqpoint{2.639063in}{4.324068in}}%
\pgfpathlineto{\pgfqpoint{2.704408in}{4.205000in}}%
\pgfpathlineto{\pgfqpoint{2.784375in}{4.079710in}}%
\pgfpathlineto{\pgfqpoint{2.796455in}{4.060000in}}%
\pgfpathlineto{\pgfqpoint{2.896764in}{3.915000in}}%
\pgfpathlineto{\pgfqpoint{2.929688in}{3.872112in}}%
\pgfpathlineto{\pgfqpoint{3.006393in}{3.770000in}}%
\pgfpathlineto{\pgfqpoint{3.075000in}{3.686523in}}%
\pgfpathlineto{\pgfqpoint{3.125231in}{3.625000in}}%
\pgfpathlineto{\pgfqpoint{3.220312in}{3.516778in}}%
\pgfpathlineto{\pgfqpoint{3.252915in}{3.480000in}}%
\pgfpathlineto{\pgfqpoint{3.365625in}{3.359902in}}%
\pgfpathlineto{\pgfqpoint{3.389582in}{3.335000in}}%
\pgfpathlineto{\pgfqpoint{3.510938in}{3.213905in}}%
\pgfpathlineto{\pgfqpoint{3.535893in}{3.190000in}}%
\pgfpathlineto{\pgfqpoint{3.656250in}{3.077532in}}%
\pgfpathlineto{\pgfqpoint{3.693107in}{3.045000in}}%
\pgfpathlineto{\pgfqpoint{3.801563in}{2.950123in}}%
\pgfpathlineto{\pgfqpoint{3.863218in}{2.900000in}}%
\pgfpathlineto{\pgfqpoint{3.946875in}{2.831541in}}%
\pgfpathlineto{\pgfqpoint{4.049207in}{2.755000in}}%
\pgfpathlineto{\pgfqpoint{4.092187in}{2.722147in}}%
\pgfpathlineto{\pgfqpoint{4.237500in}{2.622054in}}%
\pgfpathlineto{\pgfqpoint{4.257252in}{2.610000in}}%
\pgfpathlineto{\pgfqpoint{4.382812in}{2.530205in}}%
\pgfpathlineto{\pgfqpoint{4.502138in}{2.465000in}}%
\pgfpathlineto{\pgfqpoint{4.528125in}{2.450004in}}%
\pgfpathlineto{\pgfqpoint{4.673438in}{2.378840in}}%
\pgfpathlineto{\pgfqpoint{4.818750in}{2.321358in}}%
\pgfpathlineto{\pgfqpoint{4.823170in}{2.320000in}}%
\pgfpathlineto{\pgfqpoint{4.964062in}{2.273267in}}%
\pgfpathlineto{\pgfqpoint{5.109375in}{2.239092in}}%
\pgfpathlineto{\pgfqpoint{5.254688in}{2.218706in}}%
\pgfpathlineto{\pgfqpoint{5.400000in}{2.211932in}}%
\pgfusepath{stroke}%
\end{pgfscope}%
\begin{pgfscope}%
\pgfpathrectangle{\pgfqpoint{0.750000in}{0.580000in}}{\pgfqpoint{4.650000in}{4.640000in}} %
\pgfusepath{clip}%
\pgfsetbuttcap%
\pgfsetroundjoin%
\pgfsetlinewidth{1.003750pt}%
\definecolor{currentstroke}{rgb}{0.000000,0.000000,0.000000}%
\pgfsetstrokecolor{currentstroke}%
\pgfsetdash{}{0pt}%
\pgfpathmoveto{\pgfqpoint{3.657183in}{5.220000in}}%
\pgfpathlineto{\pgfqpoint{3.663596in}{5.075000in}}%
\pgfpathlineto{\pgfqpoint{3.683135in}{4.930000in}}%
\pgfpathlineto{\pgfqpoint{3.716655in}{4.785000in}}%
\pgfpathlineto{\pgfqpoint{3.765467in}{4.640000in}}%
\pgfpathlineto{\pgfqpoint{3.801563in}{4.558353in}}%
\pgfpathlineto{\pgfqpoint{3.829333in}{4.495000in}}%
\pgfpathlineto{\pgfqpoint{3.909590in}{4.350000in}}%
\pgfpathlineto{\pgfqpoint{3.946875in}{4.294029in}}%
\pgfpathlineto{\pgfqpoint{4.008741in}{4.205000in}}%
\pgfpathlineto{\pgfqpoint{4.091245in}{4.104354in}}%
\pgfusepath{stroke}%
\end{pgfscope}%
\begin{pgfscope}%
\pgfpathrectangle{\pgfqpoint{0.750000in}{0.580000in}}{\pgfqpoint{4.650000in}{4.640000in}} %
\pgfusepath{clip}%
\pgfsetbuttcap%
\pgfsetroundjoin%
\pgfsetlinewidth{1.003750pt}%
\definecolor{currentstroke}{rgb}{0.000000,0.000000,0.000000}%
\pgfsetstrokecolor{currentstroke}%
\pgfsetdash{}{0pt}%
\pgfpathmoveto{\pgfqpoint{4.395365in}{3.823048in}}%
\pgfpathlineto{\pgfqpoint{4.472034in}{3.770000in}}%
\pgfpathlineto{\pgfqpoint{4.528125in}{3.732795in}}%
\pgfpathlineto{\pgfqpoint{4.673438in}{3.652711in}}%
\pgfpathlineto{\pgfqpoint{4.736927in}{3.625000in}}%
\pgfpathlineto{\pgfqpoint{4.818750in}{3.588982in}}%
\pgfpathlineto{\pgfqpoint{4.964062in}{3.540275in}}%
\pgfpathlineto{\pgfqpoint{5.109375in}{3.506827in}}%
\pgfpathlineto{\pgfqpoint{5.254688in}{3.487331in}}%
\pgfpathlineto{\pgfqpoint{5.400000in}{3.480931in}}%
\pgfusepath{stroke}%
\end{pgfscope}%
\begin{pgfscope}%
\pgfpathrectangle{\pgfqpoint{0.750000in}{0.580000in}}{\pgfqpoint{4.650000in}{4.640000in}} %
\pgfusepath{clip}%
\pgfsetbuttcap%
\pgfsetroundjoin%
\pgfsetlinewidth{1.003750pt}%
\definecolor{currentstroke}{rgb}{0.000000,0.000000,0.000000}%
\pgfsetstrokecolor{currentstroke}%
\pgfsetdash{}{0pt}%
\pgfpathmoveto{\pgfqpoint{4.348119in}{5.220000in}}%
\pgfpathlineto{\pgfqpoint{4.359136in}{5.075000in}}%
\pgfpathlineto{\pgfqpoint{4.382812in}{4.972321in}}%
\pgfpathlineto{\pgfqpoint{4.391924in}{4.930000in}}%
\pgfpathlineto{\pgfqpoint{4.446146in}{4.785000in}}%
\pgfpathlineto{\pgfqpoint{4.528125in}{4.643927in}}%
\pgfpathlineto{\pgfqpoint{4.530563in}{4.640000in}}%
\pgfpathlineto{\pgfqpoint{4.535596in}{4.633820in}}%
\pgfusepath{stroke}%
\end{pgfscope}%
\begin{pgfscope}%
\pgfpathrectangle{\pgfqpoint{0.750000in}{0.580000in}}{\pgfqpoint{4.650000in}{4.640000in}} %
\pgfusepath{clip}%
\pgfsetbuttcap%
\pgfsetroundjoin%
\pgfsetlinewidth{1.003750pt}%
\definecolor{currentstroke}{rgb}{0.000000,0.000000,0.000000}%
\pgfsetstrokecolor{currentstroke}%
\pgfsetdash{}{0pt}%
\pgfpathmoveto{\pgfqpoint{4.833141in}{4.343951in}}%
\pgfpathlineto{\pgfqpoint{4.964062in}{4.268198in}}%
\pgfpathlineto{\pgfqpoint{5.109375in}{4.214092in}}%
\pgfpathlineto{\pgfqpoint{5.151787in}{4.205000in}}%
\pgfpathlineto{\pgfqpoint{5.254688in}{4.181374in}}%
\pgfpathlineto{\pgfqpoint{5.400000in}{4.170381in}}%
\pgfusepath{stroke}%
\end{pgfscope}%
\begin{pgfscope}%
\pgfpathrectangle{\pgfqpoint{0.750000in}{0.580000in}}{\pgfqpoint{4.650000in}{4.640000in}} %
\pgfusepath{clip}%
\pgfsetbuttcap%
\pgfsetroundjoin%
\pgfsetlinewidth{1.003750pt}%
\definecolor{currentstroke}{rgb}{0.000000,0.000000,0.000000}%
\pgfsetstrokecolor{currentstroke}%
\pgfsetdash{}{0pt}%
\pgfpathmoveto{\pgfqpoint{4.759055in}{5.220000in}}%
\pgfpathlineto{\pgfqpoint{4.778650in}{5.075000in}}%
\pgfpathlineto{\pgfqpoint{4.818750in}{4.975281in}}%
\pgfpathlineto{\pgfqpoint{4.833932in}{4.937622in}}%
\pgfusepath{stroke}%
\end{pgfscope}%
\begin{pgfscope}%
\pgfpathrectangle{\pgfqpoint{0.750000in}{0.580000in}}{\pgfqpoint{4.650000in}{4.640000in}} %
\pgfusepath{clip}%
\pgfsetbuttcap%
\pgfsetroundjoin%
\pgfsetlinewidth{1.003750pt}%
\definecolor{currentstroke}{rgb}{0.000000,0.000000,0.000000}%
\pgfsetstrokecolor{currentstroke}%
\pgfsetdash{}{0pt}%
\pgfpathmoveto{\pgfqpoint{5.128827in}{4.650407in}}%
\pgfpathlineto{\pgfqpoint{5.154753in}{4.640000in}}%
\pgfpathlineto{\pgfqpoint{5.254688in}{4.599986in}}%
\pgfpathlineto{\pgfqpoint{5.400000in}{4.580434in}}%
\pgfusepath{stroke}%
\end{pgfscope}%
\begin{pgfscope}%
\pgfpathrectangle{\pgfqpoint{0.750000in}{0.580000in}}{\pgfqpoint{4.650000in}{4.640000in}} %
\pgfusepath{clip}%
\pgfsetbuttcap%
\pgfsetroundjoin%
\pgfsetlinewidth{1.003750pt}%
\definecolor{currentstroke}{rgb}{0.000000,0.000000,0.000000}%
\pgfsetstrokecolor{currentstroke}%
\pgfsetdash{}{0pt}%
\pgfpathmoveto{\pgfqpoint{5.004357in}{5.220000in}}%
\pgfpathlineto{\pgfqpoint{5.033711in}{5.105417in}}%
\pgfusepath{stroke}%
\end{pgfscope}%
\begin{pgfscope}%
\pgfpathrectangle{\pgfqpoint{0.750000in}{0.580000in}}{\pgfqpoint{4.650000in}{4.640000in}} %
\pgfusepath{clip}%
\pgfsetbuttcap%
\pgfsetroundjoin%
\pgfsetlinewidth{1.003750pt}%
\definecolor{currentstroke}{rgb}{0.000000,0.000000,0.000000}%
\pgfsetstrokecolor{currentstroke}%
\pgfsetdash{}{0pt}%
\pgfpathmoveto{\pgfqpoint{5.330807in}{4.842858in}}%
\pgfpathlineto{\pgfqpoint{5.400000in}{4.825208in}}%
\pgfusepath{stroke}%
\end{pgfscope}%
\begin{pgfscope}%
\pgfpathrectangle{\pgfqpoint{0.750000in}{0.580000in}}{\pgfqpoint{4.650000in}{4.640000in}} %
\pgfusepath{clip}%
\pgfsetbuttcap%
\pgfsetroundjoin%
\pgfsetlinewidth{1.003750pt}%
\definecolor{currentstroke}{rgb}{0.000000,0.000000,0.000000}%
\pgfsetstrokecolor{currentstroke}%
\pgfsetdash{}{0pt}%
\pgfpathmoveto{\pgfqpoint{5.149942in}{5.220000in}}%
\pgfpathlineto{\pgfqpoint{5.226145in}{5.075000in}}%
\pgfpathlineto{\pgfqpoint{5.254688in}{5.046519in}}%
\pgfpathlineto{\pgfqpoint{5.400000in}{4.970479in}}%
\pgfusepath{stroke}%
\end{pgfscope}%
\begin{pgfscope}%
\pgfpathrectangle{\pgfqpoint{0.750000in}{0.580000in}}{\pgfqpoint{4.650000in}{4.640000in}} %
\pgfusepath{clip}%
\pgfsetbuttcap%
\pgfsetroundjoin%
\pgfsetlinewidth{1.003750pt}%
\definecolor{currentstroke}{rgb}{0.000000,0.000000,0.000000}%
\pgfsetstrokecolor{currentstroke}%
\pgfsetdash{}{0pt}%
\pgfpathmoveto{\pgfqpoint{5.400000in}{5.072791in}}%
\pgfpathlineto{\pgfqpoint{5.396325in}{5.075000in}}%
\pgfpathlineto{\pgfqpoint{5.254688in}{5.216333in}}%
\pgfpathlineto{\pgfqpoint{5.252473in}{5.220000in}}%
\pgfusepath{stroke}%
\end{pgfscope}%
\begin{pgfscope}%
\pgfpathrectangle{\pgfqpoint{0.750000in}{0.580000in}}{\pgfqpoint{4.650000in}{4.640000in}} %
\pgfusepath{clip}%
\pgfsetbuttcap%
\pgfsetroundjoin%
\pgfsetlinewidth{1.003750pt}%
\definecolor{currentstroke}{rgb}{0.000000,0.000000,0.000000}%
\pgfsetstrokecolor{currentstroke}%
\pgfsetdash{}{0pt}%
\pgfpathmoveto{\pgfqpoint{5.400000in}{5.120398in}}%
\pgfpathlineto{\pgfqpoint{5.300183in}{5.220000in}}%
\pgfusepath{stroke}%
\end{pgfscope}%
\begin{pgfscope}%
\pgfpathrectangle{\pgfqpoint{0.750000in}{0.580000in}}{\pgfqpoint{4.650000in}{4.640000in}} %
\pgfusepath{clip}%
\pgfsetbuttcap%
\pgfsetroundjoin%
\pgfsetlinewidth{1.003750pt}%
\definecolor{currentstroke}{rgb}{0.000000,0.000000,0.000000}%
\pgfsetstrokecolor{currentstroke}%
\pgfsetdash{}{0pt}%
\pgfpathmoveto{\pgfqpoint{5.400000in}{5.166798in}}%
\pgfpathlineto{\pgfqpoint{5.346683in}{5.220000in}}%
\pgfusepath{stroke}%
\end{pgfscope}%
\begin{pgfscope}%
\pgfpathrectangle{\pgfqpoint{0.750000in}{0.580000in}}{\pgfqpoint{4.650000in}{4.640000in}} %
\pgfusepath{clip}%
\pgfsetbuttcap%
\pgfsetroundjoin%
\pgfsetlinewidth{1.003750pt}%
\definecolor{currentstroke}{rgb}{0.000000,0.000000,0.000000}%
\pgfsetstrokecolor{currentstroke}%
\pgfsetdash{}{0pt}%
\pgfpathmoveto{\pgfqpoint{5.400000in}{5.213198in}}%
\pgfpathlineto{\pgfqpoint{5.393183in}{5.220000in}}%
\pgfusepath{stroke}%
\end{pgfscope}%
\begin{pgfscope}%
\pgfsetbuttcap%
\pgfsetroundjoin%
\definecolor{currentfill}{rgb}{0.000000,0.000000,0.000000}%
\pgfsetfillcolor{currentfill}%
\pgfsetlinewidth{0.501875pt}%
\definecolor{currentstroke}{rgb}{0.000000,0.000000,0.000000}%
\pgfsetstrokecolor{currentstroke}%
\pgfsetdash{}{0pt}%
\pgfsys@defobject{currentmarker}{\pgfqpoint{0.000000in}{0.000000in}}{\pgfqpoint{0.000000in}{0.055556in}}{%
\pgfpathmoveto{\pgfqpoint{0.000000in}{0.000000in}}%
\pgfpathlineto{\pgfqpoint{0.000000in}{0.055556in}}%
\pgfusepath{stroke,fill}%
}%
\begin{pgfscope}%
\pgfsys@transformshift{0.750000in}{0.580000in}%
\pgfsys@useobject{currentmarker}{}%
\end{pgfscope}%
\end{pgfscope}%
\begin{pgfscope}%
\pgfsetbuttcap%
\pgfsetroundjoin%
\definecolor{currentfill}{rgb}{0.000000,0.000000,0.000000}%
\pgfsetfillcolor{currentfill}%
\pgfsetlinewidth{0.501875pt}%
\definecolor{currentstroke}{rgb}{0.000000,0.000000,0.000000}%
\pgfsetstrokecolor{currentstroke}%
\pgfsetdash{}{0pt}%
\pgfsys@defobject{currentmarker}{\pgfqpoint{0.000000in}{-0.055556in}}{\pgfqpoint{0.000000in}{0.000000in}}{%
\pgfpathmoveto{\pgfqpoint{0.000000in}{0.000000in}}%
\pgfpathlineto{\pgfqpoint{0.000000in}{-0.055556in}}%
\pgfusepath{stroke,fill}%
}%
\begin{pgfscope}%
\pgfsys@transformshift{0.750000in}{5.220000in}%
\pgfsys@useobject{currentmarker}{}%
\end{pgfscope}%
\end{pgfscope}%
\begin{pgfscope}%
\pgftext[x=0.750000in,y=0.524444in,,top]{{\sffamily\fontsize{12.000000}{14.400000}\selectfont 0}}%
\end{pgfscope}%
\begin{pgfscope}%
\pgfsetbuttcap%
\pgfsetroundjoin%
\definecolor{currentfill}{rgb}{0.000000,0.000000,0.000000}%
\pgfsetfillcolor{currentfill}%
\pgfsetlinewidth{0.501875pt}%
\definecolor{currentstroke}{rgb}{0.000000,0.000000,0.000000}%
\pgfsetstrokecolor{currentstroke}%
\pgfsetdash{}{0pt}%
\pgfsys@defobject{currentmarker}{\pgfqpoint{0.000000in}{0.000000in}}{\pgfqpoint{0.000000in}{0.055556in}}{%
\pgfpathmoveto{\pgfqpoint{0.000000in}{0.000000in}}%
\pgfpathlineto{\pgfqpoint{0.000000in}{0.055556in}}%
\pgfusepath{stroke,fill}%
}%
\begin{pgfscope}%
\pgfsys@transformshift{1.476562in}{0.580000in}%
\pgfsys@useobject{currentmarker}{}%
\end{pgfscope}%
\end{pgfscope}%
\begin{pgfscope}%
\pgfsetbuttcap%
\pgfsetroundjoin%
\definecolor{currentfill}{rgb}{0.000000,0.000000,0.000000}%
\pgfsetfillcolor{currentfill}%
\pgfsetlinewidth{0.501875pt}%
\definecolor{currentstroke}{rgb}{0.000000,0.000000,0.000000}%
\pgfsetstrokecolor{currentstroke}%
\pgfsetdash{}{0pt}%
\pgfsys@defobject{currentmarker}{\pgfqpoint{0.000000in}{-0.055556in}}{\pgfqpoint{0.000000in}{0.000000in}}{%
\pgfpathmoveto{\pgfqpoint{0.000000in}{0.000000in}}%
\pgfpathlineto{\pgfqpoint{0.000000in}{-0.055556in}}%
\pgfusepath{stroke,fill}%
}%
\begin{pgfscope}%
\pgfsys@transformshift{1.476562in}{5.220000in}%
\pgfsys@useobject{currentmarker}{}%
\end{pgfscope}%
\end{pgfscope}%
\begin{pgfscope}%
\pgftext[x=1.476562in,y=0.524444in,,top]{{\sffamily\fontsize{12.000000}{14.400000}\selectfont 5}}%
\end{pgfscope}%
\begin{pgfscope}%
\pgfsetbuttcap%
\pgfsetroundjoin%
\definecolor{currentfill}{rgb}{0.000000,0.000000,0.000000}%
\pgfsetfillcolor{currentfill}%
\pgfsetlinewidth{0.501875pt}%
\definecolor{currentstroke}{rgb}{0.000000,0.000000,0.000000}%
\pgfsetstrokecolor{currentstroke}%
\pgfsetdash{}{0pt}%
\pgfsys@defobject{currentmarker}{\pgfqpoint{0.000000in}{0.000000in}}{\pgfqpoint{0.000000in}{0.055556in}}{%
\pgfpathmoveto{\pgfqpoint{0.000000in}{0.000000in}}%
\pgfpathlineto{\pgfqpoint{0.000000in}{0.055556in}}%
\pgfusepath{stroke,fill}%
}%
\begin{pgfscope}%
\pgfsys@transformshift{2.203125in}{0.580000in}%
\pgfsys@useobject{currentmarker}{}%
\end{pgfscope}%
\end{pgfscope}%
\begin{pgfscope}%
\pgfsetbuttcap%
\pgfsetroundjoin%
\definecolor{currentfill}{rgb}{0.000000,0.000000,0.000000}%
\pgfsetfillcolor{currentfill}%
\pgfsetlinewidth{0.501875pt}%
\definecolor{currentstroke}{rgb}{0.000000,0.000000,0.000000}%
\pgfsetstrokecolor{currentstroke}%
\pgfsetdash{}{0pt}%
\pgfsys@defobject{currentmarker}{\pgfqpoint{0.000000in}{-0.055556in}}{\pgfqpoint{0.000000in}{0.000000in}}{%
\pgfpathmoveto{\pgfqpoint{0.000000in}{0.000000in}}%
\pgfpathlineto{\pgfqpoint{0.000000in}{-0.055556in}}%
\pgfusepath{stroke,fill}%
}%
\begin{pgfscope}%
\pgfsys@transformshift{2.203125in}{5.220000in}%
\pgfsys@useobject{currentmarker}{}%
\end{pgfscope}%
\end{pgfscope}%
\begin{pgfscope}%
\pgftext[x=2.203125in,y=0.524444in,,top]{{\sffamily\fontsize{12.000000}{14.400000}\selectfont 10}}%
\end{pgfscope}%
\begin{pgfscope}%
\pgfsetbuttcap%
\pgfsetroundjoin%
\definecolor{currentfill}{rgb}{0.000000,0.000000,0.000000}%
\pgfsetfillcolor{currentfill}%
\pgfsetlinewidth{0.501875pt}%
\definecolor{currentstroke}{rgb}{0.000000,0.000000,0.000000}%
\pgfsetstrokecolor{currentstroke}%
\pgfsetdash{}{0pt}%
\pgfsys@defobject{currentmarker}{\pgfqpoint{0.000000in}{0.000000in}}{\pgfqpoint{0.000000in}{0.055556in}}{%
\pgfpathmoveto{\pgfqpoint{0.000000in}{0.000000in}}%
\pgfpathlineto{\pgfqpoint{0.000000in}{0.055556in}}%
\pgfusepath{stroke,fill}%
}%
\begin{pgfscope}%
\pgfsys@transformshift{2.929688in}{0.580000in}%
\pgfsys@useobject{currentmarker}{}%
\end{pgfscope}%
\end{pgfscope}%
\begin{pgfscope}%
\pgfsetbuttcap%
\pgfsetroundjoin%
\definecolor{currentfill}{rgb}{0.000000,0.000000,0.000000}%
\pgfsetfillcolor{currentfill}%
\pgfsetlinewidth{0.501875pt}%
\definecolor{currentstroke}{rgb}{0.000000,0.000000,0.000000}%
\pgfsetstrokecolor{currentstroke}%
\pgfsetdash{}{0pt}%
\pgfsys@defobject{currentmarker}{\pgfqpoint{0.000000in}{-0.055556in}}{\pgfqpoint{0.000000in}{0.000000in}}{%
\pgfpathmoveto{\pgfqpoint{0.000000in}{0.000000in}}%
\pgfpathlineto{\pgfqpoint{0.000000in}{-0.055556in}}%
\pgfusepath{stroke,fill}%
}%
\begin{pgfscope}%
\pgfsys@transformshift{2.929688in}{5.220000in}%
\pgfsys@useobject{currentmarker}{}%
\end{pgfscope}%
\end{pgfscope}%
\begin{pgfscope}%
\pgftext[x=2.929688in,y=0.524444in,,top]{{\sffamily\fontsize{12.000000}{14.400000}\selectfont 15}}%
\end{pgfscope}%
\begin{pgfscope}%
\pgfsetbuttcap%
\pgfsetroundjoin%
\definecolor{currentfill}{rgb}{0.000000,0.000000,0.000000}%
\pgfsetfillcolor{currentfill}%
\pgfsetlinewidth{0.501875pt}%
\definecolor{currentstroke}{rgb}{0.000000,0.000000,0.000000}%
\pgfsetstrokecolor{currentstroke}%
\pgfsetdash{}{0pt}%
\pgfsys@defobject{currentmarker}{\pgfqpoint{0.000000in}{0.000000in}}{\pgfqpoint{0.000000in}{0.055556in}}{%
\pgfpathmoveto{\pgfqpoint{0.000000in}{0.000000in}}%
\pgfpathlineto{\pgfqpoint{0.000000in}{0.055556in}}%
\pgfusepath{stroke,fill}%
}%
\begin{pgfscope}%
\pgfsys@transformshift{3.656250in}{0.580000in}%
\pgfsys@useobject{currentmarker}{}%
\end{pgfscope}%
\end{pgfscope}%
\begin{pgfscope}%
\pgfsetbuttcap%
\pgfsetroundjoin%
\definecolor{currentfill}{rgb}{0.000000,0.000000,0.000000}%
\pgfsetfillcolor{currentfill}%
\pgfsetlinewidth{0.501875pt}%
\definecolor{currentstroke}{rgb}{0.000000,0.000000,0.000000}%
\pgfsetstrokecolor{currentstroke}%
\pgfsetdash{}{0pt}%
\pgfsys@defobject{currentmarker}{\pgfqpoint{0.000000in}{-0.055556in}}{\pgfqpoint{0.000000in}{0.000000in}}{%
\pgfpathmoveto{\pgfqpoint{0.000000in}{0.000000in}}%
\pgfpathlineto{\pgfqpoint{0.000000in}{-0.055556in}}%
\pgfusepath{stroke,fill}%
}%
\begin{pgfscope}%
\pgfsys@transformshift{3.656250in}{5.220000in}%
\pgfsys@useobject{currentmarker}{}%
\end{pgfscope}%
\end{pgfscope}%
\begin{pgfscope}%
\pgftext[x=3.656250in,y=0.524444in,,top]{{\sffamily\fontsize{12.000000}{14.400000}\selectfont 20}}%
\end{pgfscope}%
\begin{pgfscope}%
\pgfsetbuttcap%
\pgfsetroundjoin%
\definecolor{currentfill}{rgb}{0.000000,0.000000,0.000000}%
\pgfsetfillcolor{currentfill}%
\pgfsetlinewidth{0.501875pt}%
\definecolor{currentstroke}{rgb}{0.000000,0.000000,0.000000}%
\pgfsetstrokecolor{currentstroke}%
\pgfsetdash{}{0pt}%
\pgfsys@defobject{currentmarker}{\pgfqpoint{0.000000in}{0.000000in}}{\pgfqpoint{0.000000in}{0.055556in}}{%
\pgfpathmoveto{\pgfqpoint{0.000000in}{0.000000in}}%
\pgfpathlineto{\pgfqpoint{0.000000in}{0.055556in}}%
\pgfusepath{stroke,fill}%
}%
\begin{pgfscope}%
\pgfsys@transformshift{4.382812in}{0.580000in}%
\pgfsys@useobject{currentmarker}{}%
\end{pgfscope}%
\end{pgfscope}%
\begin{pgfscope}%
\pgfsetbuttcap%
\pgfsetroundjoin%
\definecolor{currentfill}{rgb}{0.000000,0.000000,0.000000}%
\pgfsetfillcolor{currentfill}%
\pgfsetlinewidth{0.501875pt}%
\definecolor{currentstroke}{rgb}{0.000000,0.000000,0.000000}%
\pgfsetstrokecolor{currentstroke}%
\pgfsetdash{}{0pt}%
\pgfsys@defobject{currentmarker}{\pgfqpoint{0.000000in}{-0.055556in}}{\pgfqpoint{0.000000in}{0.000000in}}{%
\pgfpathmoveto{\pgfqpoint{0.000000in}{0.000000in}}%
\pgfpathlineto{\pgfqpoint{0.000000in}{-0.055556in}}%
\pgfusepath{stroke,fill}%
}%
\begin{pgfscope}%
\pgfsys@transformshift{4.382812in}{5.220000in}%
\pgfsys@useobject{currentmarker}{}%
\end{pgfscope}%
\end{pgfscope}%
\begin{pgfscope}%
\pgftext[x=4.382812in,y=0.524444in,,top]{{\sffamily\fontsize{12.000000}{14.400000}\selectfont 25}}%
\end{pgfscope}%
\begin{pgfscope}%
\pgfsetbuttcap%
\pgfsetroundjoin%
\definecolor{currentfill}{rgb}{0.000000,0.000000,0.000000}%
\pgfsetfillcolor{currentfill}%
\pgfsetlinewidth{0.501875pt}%
\definecolor{currentstroke}{rgb}{0.000000,0.000000,0.000000}%
\pgfsetstrokecolor{currentstroke}%
\pgfsetdash{}{0pt}%
\pgfsys@defobject{currentmarker}{\pgfqpoint{0.000000in}{0.000000in}}{\pgfqpoint{0.000000in}{0.055556in}}{%
\pgfpathmoveto{\pgfqpoint{0.000000in}{0.000000in}}%
\pgfpathlineto{\pgfqpoint{0.000000in}{0.055556in}}%
\pgfusepath{stroke,fill}%
}%
\begin{pgfscope}%
\pgfsys@transformshift{5.109375in}{0.580000in}%
\pgfsys@useobject{currentmarker}{}%
\end{pgfscope}%
\end{pgfscope}%
\begin{pgfscope}%
\pgfsetbuttcap%
\pgfsetroundjoin%
\definecolor{currentfill}{rgb}{0.000000,0.000000,0.000000}%
\pgfsetfillcolor{currentfill}%
\pgfsetlinewidth{0.501875pt}%
\definecolor{currentstroke}{rgb}{0.000000,0.000000,0.000000}%
\pgfsetstrokecolor{currentstroke}%
\pgfsetdash{}{0pt}%
\pgfsys@defobject{currentmarker}{\pgfqpoint{0.000000in}{-0.055556in}}{\pgfqpoint{0.000000in}{0.000000in}}{%
\pgfpathmoveto{\pgfqpoint{0.000000in}{0.000000in}}%
\pgfpathlineto{\pgfqpoint{0.000000in}{-0.055556in}}%
\pgfusepath{stroke,fill}%
}%
\begin{pgfscope}%
\pgfsys@transformshift{5.109375in}{5.220000in}%
\pgfsys@useobject{currentmarker}{}%
\end{pgfscope}%
\end{pgfscope}%
\begin{pgfscope}%
\pgftext[x=5.109375in,y=0.524444in,,top]{{\sffamily\fontsize{12.000000}{14.400000}\selectfont 30}}%
\end{pgfscope}%
\begin{pgfscope}%
\pgfsetbuttcap%
\pgfsetroundjoin%
\definecolor{currentfill}{rgb}{0.000000,0.000000,0.000000}%
\pgfsetfillcolor{currentfill}%
\pgfsetlinewidth{0.501875pt}%
\definecolor{currentstroke}{rgb}{0.000000,0.000000,0.000000}%
\pgfsetstrokecolor{currentstroke}%
\pgfsetdash{}{0pt}%
\pgfsys@defobject{currentmarker}{\pgfqpoint{0.000000in}{0.000000in}}{\pgfqpoint{0.055556in}{0.000000in}}{%
\pgfpathmoveto{\pgfqpoint{0.000000in}{0.000000in}}%
\pgfpathlineto{\pgfqpoint{0.055556in}{0.000000in}}%
\pgfusepath{stroke,fill}%
}%
\begin{pgfscope}%
\pgfsys@transformshift{0.750000in}{0.580000in}%
\pgfsys@useobject{currentmarker}{}%
\end{pgfscope}%
\end{pgfscope}%
\begin{pgfscope}%
\pgfsetbuttcap%
\pgfsetroundjoin%
\definecolor{currentfill}{rgb}{0.000000,0.000000,0.000000}%
\pgfsetfillcolor{currentfill}%
\pgfsetlinewidth{0.501875pt}%
\definecolor{currentstroke}{rgb}{0.000000,0.000000,0.000000}%
\pgfsetstrokecolor{currentstroke}%
\pgfsetdash{}{0pt}%
\pgfsys@defobject{currentmarker}{\pgfqpoint{-0.055556in}{0.000000in}}{\pgfqpoint{0.000000in}{0.000000in}}{%
\pgfpathmoveto{\pgfqpoint{0.000000in}{0.000000in}}%
\pgfpathlineto{\pgfqpoint{-0.055556in}{0.000000in}}%
\pgfusepath{stroke,fill}%
}%
\begin{pgfscope}%
\pgfsys@transformshift{5.400000in}{0.580000in}%
\pgfsys@useobject{currentmarker}{}%
\end{pgfscope}%
\end{pgfscope}%
\begin{pgfscope}%
\pgftext[x=0.694444in,y=0.580000in,right,]{{\sffamily\fontsize{12.000000}{14.400000}\selectfont 0}}%
\end{pgfscope}%
\begin{pgfscope}%
\pgfsetbuttcap%
\pgfsetroundjoin%
\definecolor{currentfill}{rgb}{0.000000,0.000000,0.000000}%
\pgfsetfillcolor{currentfill}%
\pgfsetlinewidth{0.501875pt}%
\definecolor{currentstroke}{rgb}{0.000000,0.000000,0.000000}%
\pgfsetstrokecolor{currentstroke}%
\pgfsetdash{}{0pt}%
\pgfsys@defobject{currentmarker}{\pgfqpoint{0.000000in}{0.000000in}}{\pgfqpoint{0.055556in}{0.000000in}}{%
\pgfpathmoveto{\pgfqpoint{0.000000in}{0.000000in}}%
\pgfpathlineto{\pgfqpoint{0.055556in}{0.000000in}}%
\pgfusepath{stroke,fill}%
}%
\begin{pgfscope}%
\pgfsys@transformshift{0.750000in}{1.305000in}%
\pgfsys@useobject{currentmarker}{}%
\end{pgfscope}%
\end{pgfscope}%
\begin{pgfscope}%
\pgfsetbuttcap%
\pgfsetroundjoin%
\definecolor{currentfill}{rgb}{0.000000,0.000000,0.000000}%
\pgfsetfillcolor{currentfill}%
\pgfsetlinewidth{0.501875pt}%
\definecolor{currentstroke}{rgb}{0.000000,0.000000,0.000000}%
\pgfsetstrokecolor{currentstroke}%
\pgfsetdash{}{0pt}%
\pgfsys@defobject{currentmarker}{\pgfqpoint{-0.055556in}{0.000000in}}{\pgfqpoint{0.000000in}{0.000000in}}{%
\pgfpathmoveto{\pgfqpoint{0.000000in}{0.000000in}}%
\pgfpathlineto{\pgfqpoint{-0.055556in}{0.000000in}}%
\pgfusepath{stroke,fill}%
}%
\begin{pgfscope}%
\pgfsys@transformshift{5.400000in}{1.305000in}%
\pgfsys@useobject{currentmarker}{}%
\end{pgfscope}%
\end{pgfscope}%
\begin{pgfscope}%
\pgftext[x=0.694444in,y=1.305000in,right,]{{\sffamily\fontsize{12.000000}{14.400000}\selectfont 5}}%
\end{pgfscope}%
\begin{pgfscope}%
\pgfsetbuttcap%
\pgfsetroundjoin%
\definecolor{currentfill}{rgb}{0.000000,0.000000,0.000000}%
\pgfsetfillcolor{currentfill}%
\pgfsetlinewidth{0.501875pt}%
\definecolor{currentstroke}{rgb}{0.000000,0.000000,0.000000}%
\pgfsetstrokecolor{currentstroke}%
\pgfsetdash{}{0pt}%
\pgfsys@defobject{currentmarker}{\pgfqpoint{0.000000in}{0.000000in}}{\pgfqpoint{0.055556in}{0.000000in}}{%
\pgfpathmoveto{\pgfqpoint{0.000000in}{0.000000in}}%
\pgfpathlineto{\pgfqpoint{0.055556in}{0.000000in}}%
\pgfusepath{stroke,fill}%
}%
\begin{pgfscope}%
\pgfsys@transformshift{0.750000in}{2.030000in}%
\pgfsys@useobject{currentmarker}{}%
\end{pgfscope}%
\end{pgfscope}%
\begin{pgfscope}%
\pgfsetbuttcap%
\pgfsetroundjoin%
\definecolor{currentfill}{rgb}{0.000000,0.000000,0.000000}%
\pgfsetfillcolor{currentfill}%
\pgfsetlinewidth{0.501875pt}%
\definecolor{currentstroke}{rgb}{0.000000,0.000000,0.000000}%
\pgfsetstrokecolor{currentstroke}%
\pgfsetdash{}{0pt}%
\pgfsys@defobject{currentmarker}{\pgfqpoint{-0.055556in}{0.000000in}}{\pgfqpoint{0.000000in}{0.000000in}}{%
\pgfpathmoveto{\pgfqpoint{0.000000in}{0.000000in}}%
\pgfpathlineto{\pgfqpoint{-0.055556in}{0.000000in}}%
\pgfusepath{stroke,fill}%
}%
\begin{pgfscope}%
\pgfsys@transformshift{5.400000in}{2.030000in}%
\pgfsys@useobject{currentmarker}{}%
\end{pgfscope}%
\end{pgfscope}%
\begin{pgfscope}%
\pgftext[x=0.694444in,y=2.030000in,right,]{{\sffamily\fontsize{12.000000}{14.400000}\selectfont 10}}%
\end{pgfscope}%
\begin{pgfscope}%
\pgfsetbuttcap%
\pgfsetroundjoin%
\definecolor{currentfill}{rgb}{0.000000,0.000000,0.000000}%
\pgfsetfillcolor{currentfill}%
\pgfsetlinewidth{0.501875pt}%
\definecolor{currentstroke}{rgb}{0.000000,0.000000,0.000000}%
\pgfsetstrokecolor{currentstroke}%
\pgfsetdash{}{0pt}%
\pgfsys@defobject{currentmarker}{\pgfqpoint{0.000000in}{0.000000in}}{\pgfqpoint{0.055556in}{0.000000in}}{%
\pgfpathmoveto{\pgfqpoint{0.000000in}{0.000000in}}%
\pgfpathlineto{\pgfqpoint{0.055556in}{0.000000in}}%
\pgfusepath{stroke,fill}%
}%
\begin{pgfscope}%
\pgfsys@transformshift{0.750000in}{2.755000in}%
\pgfsys@useobject{currentmarker}{}%
\end{pgfscope}%
\end{pgfscope}%
\begin{pgfscope}%
\pgfsetbuttcap%
\pgfsetroundjoin%
\definecolor{currentfill}{rgb}{0.000000,0.000000,0.000000}%
\pgfsetfillcolor{currentfill}%
\pgfsetlinewidth{0.501875pt}%
\definecolor{currentstroke}{rgb}{0.000000,0.000000,0.000000}%
\pgfsetstrokecolor{currentstroke}%
\pgfsetdash{}{0pt}%
\pgfsys@defobject{currentmarker}{\pgfqpoint{-0.055556in}{0.000000in}}{\pgfqpoint{0.000000in}{0.000000in}}{%
\pgfpathmoveto{\pgfqpoint{0.000000in}{0.000000in}}%
\pgfpathlineto{\pgfqpoint{-0.055556in}{0.000000in}}%
\pgfusepath{stroke,fill}%
}%
\begin{pgfscope}%
\pgfsys@transformshift{5.400000in}{2.755000in}%
\pgfsys@useobject{currentmarker}{}%
\end{pgfscope}%
\end{pgfscope}%
\begin{pgfscope}%
\pgftext[x=0.694444in,y=2.755000in,right,]{{\sffamily\fontsize{12.000000}{14.400000}\selectfont 15}}%
\end{pgfscope}%
\begin{pgfscope}%
\pgfsetbuttcap%
\pgfsetroundjoin%
\definecolor{currentfill}{rgb}{0.000000,0.000000,0.000000}%
\pgfsetfillcolor{currentfill}%
\pgfsetlinewidth{0.501875pt}%
\definecolor{currentstroke}{rgb}{0.000000,0.000000,0.000000}%
\pgfsetstrokecolor{currentstroke}%
\pgfsetdash{}{0pt}%
\pgfsys@defobject{currentmarker}{\pgfqpoint{0.000000in}{0.000000in}}{\pgfqpoint{0.055556in}{0.000000in}}{%
\pgfpathmoveto{\pgfqpoint{0.000000in}{0.000000in}}%
\pgfpathlineto{\pgfqpoint{0.055556in}{0.000000in}}%
\pgfusepath{stroke,fill}%
}%
\begin{pgfscope}%
\pgfsys@transformshift{0.750000in}{3.480000in}%
\pgfsys@useobject{currentmarker}{}%
\end{pgfscope}%
\end{pgfscope}%
\begin{pgfscope}%
\pgfsetbuttcap%
\pgfsetroundjoin%
\definecolor{currentfill}{rgb}{0.000000,0.000000,0.000000}%
\pgfsetfillcolor{currentfill}%
\pgfsetlinewidth{0.501875pt}%
\definecolor{currentstroke}{rgb}{0.000000,0.000000,0.000000}%
\pgfsetstrokecolor{currentstroke}%
\pgfsetdash{}{0pt}%
\pgfsys@defobject{currentmarker}{\pgfqpoint{-0.055556in}{0.000000in}}{\pgfqpoint{0.000000in}{0.000000in}}{%
\pgfpathmoveto{\pgfqpoint{0.000000in}{0.000000in}}%
\pgfpathlineto{\pgfqpoint{-0.055556in}{0.000000in}}%
\pgfusepath{stroke,fill}%
}%
\begin{pgfscope}%
\pgfsys@transformshift{5.400000in}{3.480000in}%
\pgfsys@useobject{currentmarker}{}%
\end{pgfscope}%
\end{pgfscope}%
\begin{pgfscope}%
\pgftext[x=0.694444in,y=3.480000in,right,]{{\sffamily\fontsize{12.000000}{14.400000}\selectfont 20}}%
\end{pgfscope}%
\begin{pgfscope}%
\pgfsetbuttcap%
\pgfsetroundjoin%
\definecolor{currentfill}{rgb}{0.000000,0.000000,0.000000}%
\pgfsetfillcolor{currentfill}%
\pgfsetlinewidth{0.501875pt}%
\definecolor{currentstroke}{rgb}{0.000000,0.000000,0.000000}%
\pgfsetstrokecolor{currentstroke}%
\pgfsetdash{}{0pt}%
\pgfsys@defobject{currentmarker}{\pgfqpoint{0.000000in}{0.000000in}}{\pgfqpoint{0.055556in}{0.000000in}}{%
\pgfpathmoveto{\pgfqpoint{0.000000in}{0.000000in}}%
\pgfpathlineto{\pgfqpoint{0.055556in}{0.000000in}}%
\pgfusepath{stroke,fill}%
}%
\begin{pgfscope}%
\pgfsys@transformshift{0.750000in}{4.205000in}%
\pgfsys@useobject{currentmarker}{}%
\end{pgfscope}%
\end{pgfscope}%
\begin{pgfscope}%
\pgfsetbuttcap%
\pgfsetroundjoin%
\definecolor{currentfill}{rgb}{0.000000,0.000000,0.000000}%
\pgfsetfillcolor{currentfill}%
\pgfsetlinewidth{0.501875pt}%
\definecolor{currentstroke}{rgb}{0.000000,0.000000,0.000000}%
\pgfsetstrokecolor{currentstroke}%
\pgfsetdash{}{0pt}%
\pgfsys@defobject{currentmarker}{\pgfqpoint{-0.055556in}{0.000000in}}{\pgfqpoint{0.000000in}{0.000000in}}{%
\pgfpathmoveto{\pgfqpoint{0.000000in}{0.000000in}}%
\pgfpathlineto{\pgfqpoint{-0.055556in}{0.000000in}}%
\pgfusepath{stroke,fill}%
}%
\begin{pgfscope}%
\pgfsys@transformshift{5.400000in}{4.205000in}%
\pgfsys@useobject{currentmarker}{}%
\end{pgfscope}%
\end{pgfscope}%
\begin{pgfscope}%
\pgftext[x=0.694444in,y=4.205000in,right,]{{\sffamily\fontsize{12.000000}{14.400000}\selectfont 25}}%
\end{pgfscope}%
\begin{pgfscope}%
\pgfsetbuttcap%
\pgfsetroundjoin%
\definecolor{currentfill}{rgb}{0.000000,0.000000,0.000000}%
\pgfsetfillcolor{currentfill}%
\pgfsetlinewidth{0.501875pt}%
\definecolor{currentstroke}{rgb}{0.000000,0.000000,0.000000}%
\pgfsetstrokecolor{currentstroke}%
\pgfsetdash{}{0pt}%
\pgfsys@defobject{currentmarker}{\pgfqpoint{0.000000in}{0.000000in}}{\pgfqpoint{0.055556in}{0.000000in}}{%
\pgfpathmoveto{\pgfqpoint{0.000000in}{0.000000in}}%
\pgfpathlineto{\pgfqpoint{0.055556in}{0.000000in}}%
\pgfusepath{stroke,fill}%
}%
\begin{pgfscope}%
\pgfsys@transformshift{0.750000in}{4.930000in}%
\pgfsys@useobject{currentmarker}{}%
\end{pgfscope}%
\end{pgfscope}%
\begin{pgfscope}%
\pgfsetbuttcap%
\pgfsetroundjoin%
\definecolor{currentfill}{rgb}{0.000000,0.000000,0.000000}%
\pgfsetfillcolor{currentfill}%
\pgfsetlinewidth{0.501875pt}%
\definecolor{currentstroke}{rgb}{0.000000,0.000000,0.000000}%
\pgfsetstrokecolor{currentstroke}%
\pgfsetdash{}{0pt}%
\pgfsys@defobject{currentmarker}{\pgfqpoint{-0.055556in}{0.000000in}}{\pgfqpoint{0.000000in}{0.000000in}}{%
\pgfpathmoveto{\pgfqpoint{0.000000in}{0.000000in}}%
\pgfpathlineto{\pgfqpoint{-0.055556in}{0.000000in}}%
\pgfusepath{stroke,fill}%
}%
\begin{pgfscope}%
\pgfsys@transformshift{5.400000in}{4.930000in}%
\pgfsys@useobject{currentmarker}{}%
\end{pgfscope}%
\end{pgfscope}%
\begin{pgfscope}%
\pgftext[x=0.694444in,y=4.930000in,right,]{{\sffamily\fontsize{12.000000}{14.400000}\selectfont 30}}%
\end{pgfscope}%
\begin{pgfscope}%
\pgfsetbuttcap%
\pgfsetroundjoin%
\pgfsetlinewidth{1.003750pt}%
\definecolor{currentstroke}{rgb}{0.000000,0.000000,0.000000}%
\pgfsetstrokecolor{currentstroke}%
\pgfsetdash{}{0pt}%
\pgfpathmoveto{\pgfqpoint{0.750000in}{5.220000in}}%
\pgfpathlineto{\pgfqpoint{5.400000in}{5.220000in}}%
\pgfusepath{stroke}%
\end{pgfscope}%
\begin{pgfscope}%
\pgfsetbuttcap%
\pgfsetroundjoin%
\pgfsetlinewidth{1.003750pt}%
\definecolor{currentstroke}{rgb}{0.000000,0.000000,0.000000}%
\pgfsetstrokecolor{currentstroke}%
\pgfsetdash{}{0pt}%
\pgfpathmoveto{\pgfqpoint{5.400000in}{0.580000in}}%
\pgfpathlineto{\pgfqpoint{5.400000in}{5.220000in}}%
\pgfusepath{stroke}%
\end{pgfscope}%
\begin{pgfscope}%
\pgfsetbuttcap%
\pgfsetroundjoin%
\pgfsetlinewidth{1.003750pt}%
\definecolor{currentstroke}{rgb}{0.000000,0.000000,0.000000}%
\pgfsetstrokecolor{currentstroke}%
\pgfsetdash{}{0pt}%
\pgfpathmoveto{\pgfqpoint{0.750000in}{0.580000in}}%
\pgfpathlineto{\pgfqpoint{5.400000in}{0.580000in}}%
\pgfusepath{stroke}%
\end{pgfscope}%
\begin{pgfscope}%
\pgfsetbuttcap%
\pgfsetroundjoin%
\pgfsetlinewidth{1.003750pt}%
\definecolor{currentstroke}{rgb}{0.000000,0.000000,0.000000}%
\pgfsetstrokecolor{currentstroke}%
\pgfsetdash{}{0pt}%
\pgfpathmoveto{\pgfqpoint{0.750000in}{0.580000in}}%
\pgfpathlineto{\pgfqpoint{0.750000in}{5.220000in}}%
\pgfusepath{stroke}%
\end{pgfscope}%
\begin{pgfscope}%
\pgftext[x=0.773968in,y=0.927869in,left,base,rotate=330.574181]{{\sffamily\fontsize{9.000000}{10.800000}\selectfont 0.400}}%
\end{pgfscope}%
\begin{pgfscope}%
\pgftext[x=0.974606in,y=1.090366in,left,base,rotate=324.659360]{{\sffamily\fontsize{9.000000}{10.800000}\selectfont 0.480}}%
\end{pgfscope}%
\begin{pgfscope}%
\pgftext[x=1.226307in,y=1.377326in,left,base,rotate=325.704929]{{\sffamily\fontsize{9.000000}{10.800000}\selectfont 0.560}}%
\end{pgfscope}%
\begin{pgfscope}%
\pgftext[x=1.686085in,y=1.818355in,left,base,rotate=323.622940]{{\sffamily\fontsize{9.000000}{10.800000}\selectfont 0.640}}%
\end{pgfscope}%
\begin{pgfscope}%
\pgftext[x=2.633931in,y=2.268353in,left,base,rotate=318.272553]{{\sffamily\fontsize{9.000000}{10.800000}\selectfont 0.720}}%
\end{pgfscope}%
\begin{pgfscope}%
\pgftext[x=2.353933in,y=4.941708in,left,base,rotate=289.773755]{{\sffamily\fontsize{9.000000}{10.800000}\selectfont 0.800}}%
\end{pgfscope}%
\begin{pgfscope}%
\pgftext[x=4.071813in,y=4.029111in,left,base,rotate=324.497057]{{\sffamily\fontsize{9.000000}{10.800000}\selectfont 0.880}}%
\end{pgfscope}%
\begin{pgfscope}%
\pgftext[x=4.510123in,y=4.551072in,left,base,rotate=322.762121]{{\sffamily\fontsize{9.000000}{10.800000}\selectfont 0.960}}%
\end{pgfscope}%
\begin{pgfscope}%
\pgftext[x=4.800772in,y=4.845488in,left,base,rotate=322.744450]{{\sffamily\fontsize{9.000000}{10.800000}\selectfont 1.040}}%
\end{pgfscope}%
\begin{pgfscope}%
\pgftext[x=4.990663in,y=5.001752in,left,base,rotate=325.901487]{{\sffamily\fontsize{9.000000}{10.800000}\selectfont 1.120}}%
\end{pgfscope}%
\end{pgfpicture}%
\makeatother%
\endgroup%
}
    \caption{Contour plot of the pressure difference calulated using equation \eqref{eq:differential-flow-discretzed-dimensionless} with $M=32$, solved using the program listing \ref{lst:peaceman77_solver}. The injector is placed in the upper right corner; the producer is placed in the lower left corner. The pressure difference is calculated relative to the block containing the producer.}
    \label{fig:pressure_drop_contour}
\end{figure}

\begin{figure}[htbp]
    \centering
    \includegraphics[]{figures/plots/peaceman77-regression.pdf}
    \caption{Plot of numerical solution of pressure plotted agains radius from producing well. Calculations are done using the program in Listing~\ref{lst:peaceman77_solver}. The equation is solved on a $10\times 10$ grid like the one shown in Figure~\ref{fig:peaceman-grid}.}
    \label{fig:peaceman77_pressure_vs_radius}
\end{figure}

% subsubsection approximate_calculation_of_equivalent_radius (end)





\subsubsection{Exact calculation of equivalent radius} % (fold)
\label{ssub:exact_calculation_of_equivalent_radius}
It is also possible to calculate $r_{eq}$ exactly without using regression (as shown in Figure~\ref{fig:peaceman77_pressure_vs_radius}) \cite{Peaceman1978Interpretation}. This is done by solving the numerical dimensionless steady-state pressure equation \eqref{eq:differential-flow-discretized} to acquire the pressure difference between the producer and the injector, $\Delta p$. We then use the equation for the pressure drop between injector and producer in a repeated five-spot pattern \cite{Muskat1946Flow}
\begin{equation}
    \label{eq:muskat-pressure-drop}
    \Delta p = \frac{q\mu}{\pi k h} \left( \ln \frac{d}{r_w} - 0.6190 \right)
\end{equation}
\nomenclature{$\Delta p$}{pressure difference}
\nomenclature{$d$}{diagonal distance between producer and injector}
where $d$ is the diagonal distance between producer and injector
\begin{equation}
    \label{eq:muskat-diagonal}
    d = \sqrt{2} M \Delta x.
\end{equation}
Combining \eqref{eq:muskat-pressure-drop} and \eqref{eq:muskat-diagonal}, inserting $\Delta p = p_{M,M} - p_{0,0}$, replacing $r_w$ with $r_{eq}$ and solving for dimensionless radius yields
\begin{equation}
    \label{eq:peaceman77-eqrad-exact}
    \frac{r_{eq}}{\Delta x} = \sqrt{2}M \exp\left[- \frac{\pi k h}{q\mu} \left( p_{M,M} - p_{0,0} \right) - 0.6190 \right]
\end{equation}
or, using dimensionless pressure,
\begin{equation}
    \frac{r_{eq}}{\Delta x} = \sqrt{2}M \exp\left[- \pi \left( \Phi_{M,M} - \Phi_{0,0} \right) - 0.6190 \right].
\end{equation}

Results computed using the program \ref{lst:peaceman77_solver} for various grid sizes are shown in Table~\ref{tbl:peaceman-results}.

% subsubsection exact_calculation_of_equivalent_radius (end)

\begin{table}
    \centering
    \caption{Computed equivalent radius of well-block $r_{eq}$ for various values of $M\times M$ grid sizes. $\Phi_{M,M} - \Phi_{0,0}$ is the dimensionless pressure drop between producer and injector, calculated using using the program in Listing~\ref{lst:peaceman77_solver}; $r_{eq}/\Delta x$ (exact) refers to the solution using equation \eqref{eq:peaceman77-eqrad-exact}; $r_{eq} / \Delta x$ (regression) refers to the point where the regression line crosses the $x$-axis (see Figure~\ref{fig:peaceman77_pressure_vs_radius}).}
    \begin{tabular}{rccc}
        \toprule
        $M$ & $\Phi_{M,M} - \Phi_{o,o}$ & $r_{eq}/\Delta x$ (exact) & $r_{eq} / \Delta x$ (regression)\\
        \midrule
        3   & 0.78571 & 0.1936 & 0.297 \\
        5   & 0.94346 & 0.1965 & 0.225 \\
        10  & 1.16209 & 0.1978 & 0.212 \\
        15  & 1.29078 & 0.1980 & 0.210 \\
        20  & 1.38222 & 0.1981 & 0.208 \\
        32  & 1.53173 & 0.1981 & 0.207 \\
        50  & 1.67375 & 0.1982 & 0.207 \\
        100 & 1.89436 & 0.1982 & 0.208 \\
        \bottomrule
    \end{tabular}
    \label{tbl:peaceman-results}
\end{table}

% subsection equivalent_wellbore_radius_from_numerical_solution_of_differential_flow_equation (end)

\subsection{Equivalent wellbore radius from Schlumberger ECL100 simulation results} % (fold)
\label{sub:equivalent_wellbore_radius_from_schlumberger_ecl100_simulation_results}

% subsection equivalent_wellbore_radius_from_schlumberger_ecl100_simulation_results (end)

\subsection{Equivalent wellbore radius from Sintef MRST simulation results} % (fold)
\label{sub:equivalent_wellbore_radius_from_sintef_mrst_simulation_results}

% subsection equivalent_wellbore_radius_from_sintef_mrst_simulation_results (end)


% section results (end)
