% -*- root: ../project.tex -*-

\section{Theory} % (fold)
\label{sec:theory}
This section contains a thorough review of the equations discussed in Section~\ref{sec:introduction}. This includes derivations of the expressions for the well index and equivalent wellbore radius, as well as  the discretized form of the  diffusivity equation originally used by Peaceman to calculate the value of the equivalent wellbore radius.

\subsection{Peaceman's Well Model And The Well Index} % (fold)
\label{sub:well_indices}
For a single well in a two-dimensional reservoir reservoir, Darcy's law for radial one-phase flow in the absence of skin effect may be expressed as \cite{Dake1978Developments}
\begin{equation}
    \label{eq:darcy}
    q = \frac{kA}{\mu} \frac{\mathrm{d}p}{\mathrm{d}r}
\end{equation}
which after separating variables, integrating from $p_{wf}$ to $p$ and solving for $p$ gives us the radial flow equation
\begin{equation}
    \label{eq:dake-radial-pressure}
    p(r) = p_{wf} + \frac{q\mu}{2\pi kh} \ln{\frac{r}{r_w}}
\end{equation}

Substituting $r=r_{eq}$ into equation \eqref{eq:dake-radial-pressure}, where $r_{eq}$ is the equivalent wellbore radius, the radius at which the steady-state flowing pressure for the actual well is equal to the calculated pressure for the well block, $p_{wb}$, gives
\begin{equation}
    \label{eq:peaceman-pwb}
    p_{wb} = p_{wf} + \frac{q\mu}{2\pi kh} \ln{\frac{r_{eq}}{r_w}}
\end{equation}
which is the expression commonly associated with Peaceman's well model \cite{Peaceman1978Interpretation}.

Solving \eqref{eq:peaceman-pwb} for $q$ gives
\begin{equation}
    \label{eq:peaceman-model-q}
    q = \frac{2\pi k h / \mu}{\ln (r_{eq}/{r_w})} \left( p_{wb} - p_{wf} \right).
\end{equation}
Taking the terms of this equation associated with well block geometry, well geometry and rock properties gives the commonly used expression for the well index
\begin{equation}
    \label{eq:peaceman-wi2}
    WI = \frac{2\pi kh}{\ln{r_{eq}/r_w}}
\end{equation}

Substituting \eqref{eq:peaceman-wi2} back into \eqref{eq:peaceman-model-q} yields the expression commonly used to calculate the flow from a well block into the well it contains
\begin{equation}
    q = \frac{WI}{\mu} \left( p_{wb} - p_{wf} \right)
\end{equation}


\nomenclature{$h$}{formation thickness}
\nomenclature{$k$}{permeability}
\nomenclature{$r$}{radius}
\nomenclature{$r_w$}{wellbore radius}
\nomenclature{$r_{eq}$}{equivalent well block radius}
\nomenclature{$A$}{cross-sectional area}
\nomenclature{$q$}{volumetric flow rate $[\mathrm{m^3/s}]$}
\nomenclature{$\mu$}{viscosity}
\nomenclature{$WI$}{well index}
\nomenclature{$p$}{pressure}
\nomenclature{$p_{wf}$}{flowing well pressure}
\nomenclature{$p_{wb}$}{well block pressure}
% subsection well_indices (end)





\subsection{Equivalent Well Block Radius} % (fold)
\label{sub:equivalent_wellblock_radius}
The correct value for the equivalent well block radius $r_{eq}$ may be found by examining a numerical solution to the diffusivity equation on an $M\times M$ square grid like the one shown in Figure~\ref{fig:peaceman-grid}.
\nomenclature{$M$}{number of blocks on each side of computing grid}

\begin{figure}[htb]
    \centering
    \includegraphics[]{figures/grids/peaceman77-10x10.pdf}
    \caption{$10\times 10$ grid used with the five-point stencil. The circle in the lower-left corner represents a producing well; the dot in the upper-right corner represents an injector.}
    \label{fig:peaceman-grid}
\end{figure}



\subsubsection{Derivation Of The Discretized Diffusivity Equation} % (fold)
\label{ssub:derivation}
To derive the discrete form of the diffusivity equation, we  start with the continuity equation for two-dimensional, single-phase, compressible flow \cite{Peaceman1977Fundamentals}
\begin{equation}
    \label{eq:continuity-2d}
    - \frac{\partial}{\partial x} \left( h\rho v_x \right) - \frac{\partial}{\partial y} \left( h\rho v_y \right) + hQ = h \frac{\partial }{\partial t} \left( \phi \rho \right)
\end{equation}
\nomenclature{$\rho$}{density}
\nomenclature{$v_x$}{flow velocity in $x$-direction}
\nomenclature{$v_y$}{flow velocity in $y$-direction}
\nomenclature{$\phi$}{porosity}
\nomenclature{$Q$}{mass rate of injection per unit volume of reservoir $[\mathrm{kg/m^3s}]$}
where $Q$ is the mass rate of injection or production per unit volume of reservoir
\begin{equation}
    \label{eq:mass-rate}
    Q = \frac{q\rho}{Ah}.
\end{equation}
Substituting equation \eqref{eq:mass-rate} and applying steady state conditions reduces equation \eqref{eq:continuity-2d} to
\begin{equation}
    - \frac{\partial}{\partial x} \left( h\rho v_x \right) - \frac{\partial}{\partial y} \left( h\rho v_y \right) + \frac{q\rho}{A} = 0
\end{equation}
substituting Darcy's law in the $x$- and $y$-directions \cite{Peaceman1977Fundamentals}
\begin{subequations}
    \begin{equation}
        v_x = -\frac{k}{\mu} \frac{\partial p}{\partial x}
    \end{equation}
    \begin{equation}
        v_y = -\frac{k}{\mu} \frac{\partial p}{\partial y}
    \end{equation}
\end{subequations}
and taking $h,\rho,k$ and $\mu$ as constants gives us the diffusivity equation for 2-dimensional incompressible steady-state flow in a homogeneous reservoir
\begin{equation}
    \label{eq:differential-flow}
    \frac{khA}{\mu} \frac{\partial^2 p}{\partial x^2}  + \frac{khA}{\mu} \frac{\partial^2 p}{\partial y^2} = -q
\end{equation}

To acquire a discretized form of equation \eqref{eq:differential-flow}, central Taylor expansion is applied to the pressure differential, giving
\begin{subequations}
    \begin{equation}
        \frac{\partial^2 p_{i,j}}{\partial x^2} = \frac{p_{i-1,j} -2p_{i,j} + p_{i+1,j}}{\Delta x^2} + O(\Delta x^2)
    \end{equation}
    \begin{equation}
        \frac{\partial^2 p_{i,j}}{\partial y^2} = \frac{p_{i,j-1} -2p_{i,j} + p_{i,j+1}}{\Delta y^2} + O(\Delta y^2)
    \end{equation}
\end{subequations}
where $O(\Delta x^2)$ and $O(\Delta x^2)$ are error-terms and subscript $i,j$ indicate the coordinates on the computational grid. Substituting this and $A=\Delta x \Delta y$ into \eqref{eq:differential-flow}, and taking $\Delta x = \Delta y$, gives us the discretized form of the diffusivity equation on a square grid
\begin{equation}
    \label{eq:differential-flow-discretized}
    \frac{kh}{\mu} \left( p_{i-1,j} + p_{i+1,j} + p_{i,j-1} + p_{i,j+1} -4p_{i,j} \right) = -q_{i,j}
\end{equation}
\nomenclature{$\Delta x$}{grid spacing in $x$-direction}
\nomenclature{$\Delta y$}{grid spacing in $y$-direction}
\nomenclature{$p_{i,j}$}{pressure in block $(i,j)$}

We then define dimensionless pressure as
\begin{equation}
    \label{eq:dimensionless-pressure}
    \Phi = \frac{kh}{q\mu} p
\end{equation}
to get the dimensionless form of equation \eqref{eq:differential-flow-discretized} \cite{Peaceman1978Interpretation}
\begin{equation}
    \label{eq:differential-flow-discretzed-dimensionless}
     \Phi_{i-1,j} + \Phi_{i+1,j} + \Phi_{i,j-1}  + \Phi_{i,j+1} -4 \Phi_{i,j}  = \delta_{i,j}
\end{equation}
where the production and injection wells in $(0,0)$ and $(M,M)$ are represented by the conditions
\begin{subequations}
    \label{eq:well-conditions}
    \begin{equation}
        \delta_{i,j} = 0 \;\mathrm{for}\; i,j \neq 0,0 \;\mathrm{or}\; M,M,
    \end{equation}    
    \begin{equation}
        \delta_{0,0} = 1,
    \end{equation}
    \begin{equation}
        \delta_{M,M} = -1.
    \end{equation}
\end{subequations}

\nomenclature{$\Psi$}{dimensionless pressure}
\nomenclature{$\Psi_{i,j}$}{dimensionless pressure in block $(i,j)$}

The following reflection conditions are applied:
\begin{subequations}
    \label{eq:reflection-conditions}
    \begin{equation}
        \begin{rcases}
            p_{-1,j} &= p_{1,j} \\
            p_{M+1,j} &= p_{M-1,j}
        \end{rcases} \; \mathrm{for} \; 0\leq j \leq M
    \end{equation}    
    \begin{equation}
        \begin{rcases}
            p_{i,-1} &= p_{i,1} \\
            p_{i,M+1} &= p_{i,M-1}
        \end{rcases} \; \mathrm{for} \; 0\leq i \leq M
    \end{equation}
\end{subequations}

Combined, these reflection conditions constitute a no-flow boundary through the center of the edge-blocks. 

Equation~\eqref{eq:differential-flow-discretzed-dimensionless}, along with conditions \eqref{eq:well-conditions} and \eqref{eq:reflection-conditions}, gives an $(M+1)\times (M+1)$ equation system
\begin{equation}
    \label{eq:equation-system-matrix}
    \mathbf{Ap}=\boldsymbol{\delta},
\end{equation}
which may be solved directly using Gaussian elimination. The matrix structure is sketched in Figure~\ref{fig:matrix-structure}.

\begin{figure}[htb]
    \centering
    \includegraphics[width=0.95\textwidth]{figures/other/eq-system-structure_m4.pdf}
    \caption{Sketch of the equation system \eqref{eq:equation-system-matrix} resulting from equation \eqref{eq:differential-flow-discretzed-dimensionless} with $M=4$.}
    \label{fig:matrix-structure}
\end{figure}
% subsubsection derivation (end)


\subsubsection{Approximate Calculation Of Equivalent Radius} % (fold)
\label{ssub:approximate_calculation_of_equivalent_radius}
\paragraph{Graphical Approximation} % (fold)
\label{par:graphical_approximation}
One method for acquiring the correct value of the equivalent radius is by creating a semi-logarithmic plot of the pressure difference $(\Phi_{i,j}-\Phi_{0,0})$ against the radius from the producer $r/\Delta x=\sqrt{i^2+j^2}$ and extrapolate the regression line down to the horizontal line where $(\Phi_{i,j}-\Phi_{0,0})=0$. One such plot is shown in Figure~\ref{fig:peaceman77_pressure_vs_radius}. We then see that
\begin{equation}
    r_{eq} = 0.2 \Delta x
\end{equation}
This result is used by Peaceman for a new interpretation of the well block pressure: it is the steady-state flowing pressure at $r=r_{eq}=0.2 \Delta x$ \cite{Peaceman1978Interpretation}.

\begin{figure}[htb]
    \centering
    \includegraphics[]{figures/plots/peaceman77-regression.pdf}
    \caption{Plot of numerical solution to equation~\eqref{eq:differential-flow-discretzed-dimensionless} plotted against radius from producing well, along with the linear regression line. This plot is made from the solution on a $10\times 10$ grid.}
    \label{fig:peaceman77_pressure_vs_radius}
\end{figure}
% paragraph graphical_approximation (end)

\paragraph{Semi-analytical Approximation} % (fold)
\label{par:semi_analytical_approximation}

We can also see from Figure~\ref{fig:peaceman77_pressure_vs_radius} that the pressure calculated for the block adjacent to the well block, i.e. at $r/\Delta x=1$, is almost exactly on the straight line. Thus we can substitute $r=\Delta x$ into the radial flow equation \eqref{eq:dake-radial-pressure} to obtain the equation for the pressure in the adjacent block (using the block numbering shown in Figure~\ref{fig:peaceman-block-numbering})
\begin{equation}
    \label{eq:peaceman-pressure-adjacent}
    p_1 = p_{wb} + \frac{q\mu}{2\pi kh} \ln{\frac{\Delta x}{r_{eq}}}
\end{equation}
We now use equation \eqref{eq:differential-flow-discretized},
\begin{equation}
    \label{eq:peaceman-pressure-adjacent-discrete}
    \frac{kh}{\mu} \left( p_1 + p_2 + p_3 + p_4 -4p_{wb}  \right) = q
\end{equation}
where, by symmetry,
\begin{equation}
    \label{eq:peaceman-adjacent-symmetry}
    p_1=p_2=p_3=p_4.
\end{equation}
Combining equations \eqref{eq:peaceman-pressure-adjacent}, \eqref{eq:peaceman-pressure-adjacent-discrete} and \eqref{eq:peaceman-adjacent-symmetry} gives us
\begin{equation}
    \ln{\frac{\Delta x}{r_{eq}}} = \frac{\pi}{2}
\end{equation}
or
\begin{equation}
    r_{eq} = \Delta x \exp{\left( -\frac{\pi}{2} \right)} = 0.208 \Delta x
\end{equation}
which fits well with value of $r_{eq}$ acquired from inspecting the intersection of the regression line \cite{Peaceman1978Interpretation}.
% paragraph semi_analytical_approximation (end)

\begin{figure}[htb]
    \centering
    \includegraphics[]{figures/grids/peaceman77-blocks.pdf}
    \caption{Illustration of the block numbering used in equation \eqref{eq:peaceman-pressure-adjacent-discrete}.}
    \label{fig:peaceman-block-numbering}
\end{figure}
% subsubsection approximate_calculation_of_equivalent_radius (end)







\subsubsection{Exact Calculation Of Equivalent Radius} % (fold)
\label{ssub:exact_calculation_of_equivalent_radius}
It is also possible to calculate $r_{eq}$ exactly without using the regression method discussed in Section~\ref{ssub:approximate_calculation_of_equivalent_radius} \cite{Peaceman1978Interpretation}. This is done by solving the numerical dimensionless steady-state pressure equation \eqref{eq:differential-flow-discretized} to acquire the pressure difference between the producer and the injector, $p_{M,M}-p_{0,0}=\Delta p$. We then use the equation for the pressure drop between injector and producer in a repeated five-spot pattern \cite{Muskat1946Flow}
\begin{equation}
    \label{eq:muskat-pressure-drop}
    \Delta p = \frac{q\mu}{\pi k h} \left( \ln \frac{d}{r_w} - 0.6190 \right)
\end{equation}
\nomenclature{$\Delta p$}{pressure difference}
\nomenclature{$d$}{diagonal distance between producer and injector}
where $d$ is the diagonal distance between producer and injector
\begin{equation}
    \label{eq:muskat-diagonal}
    d = \sqrt{2} M \Delta x.
\end{equation}
Combining \eqref{eq:muskat-pressure-drop} and \eqref{eq:muskat-diagonal}, inserting $\Delta p = p_{M,M} - p_{0,0}$, replacing $r_w$ with $r_{eq}$ and solving for dimensionless radius yields
\begin{equation}
    \label{eq:peaceman77-eqrad-exact}
    \frac{r_{eq}}{\Delta x} = \sqrt{2}M \exp\left[- \frac{\pi k h}{q\mu} \left( p_{M,M} - p_{0,0} \right) - 0.6190 \right]
\end{equation}
or, using dimensionless pressure,
\begin{equation}
    \label{eq:peaceman77-eqrad-exact-dimensionless}
    \frac{r_{eq}}{\Delta x} = \sqrt{2}M \exp\left[- \pi \left( \Phi_{M,M} - \Phi_{0,0} \right) - 0.6190 \right].
\end{equation}

Note that Peaceman later argued that the constant term $0.6190$ in Muskat's equation \eqref{eq:muskat-pressure-drop} is inaccurate \cite{Peaceman1983Interpretation}. This is because Muskat in his derivation considered an infinite series to be negligible. In his 1983 paper, Peaceman includes the five first terms of this series, thus obtaining the more accurate value $0.61738575$. This causes a difference in the fourth significant digits of the results. This more precise value is not used here in order to replicate the results of Peaceman's 1978 paper as closely as possible.

% subsubsection exact_calculation_of_equivalent_radius (end)

% subsection equivalent_wellblock_radius (end)


% section theory (end)
