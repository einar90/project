% -*- root: ../project.tex -*-

\section{Introduction} % (fold)
\label{sec:introduction}
In reservoir simulation, a number of differential equations are used to describe the flow of fluids. These equations can only be solved analytically for very simple systems involving homogeneous reservoirs with regular boundaries \cite{Peaceman1977Fundamentals}. Because of this, we need to create discrete models for the reservoir and solve the flow equations numerically.

The reservoir is discretized by dividing it into \emph{grid blocks}. Each block has a set of properties (porosity, permeability, saturations) etc., describing the corresponding area of the reservoir. This is illustrated in Figure~\ref{fig:grid-discretization}. This discretization allows us to flow within the model using a set of easily solved linear equations, each of which describe the state of each block and the flow into- or out of it.

\begin{figure}[htbp]
    \centering
    \includegraphics[]{figures/grids/3d_grid_4x3x2.pdf}
    \caption{Illustration of a continuous reservoir vs. a discretized reservoir. In the continuous reservoir the properties are continuous functions in the spatial dimensions; in the discretized reservoir, each block has a set of properties that does not vary within the it.}
    \label{fig:grid-discretization}
\end{figure}

% section introduction (end)
